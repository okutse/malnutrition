%Version 2.1 April 2023
% See section 11 of the User Manual for version history
%
%%%%%%%%%%%%%%%%%%%%%%%%%%%%%%%%%%%%%%%%%%%%%%%%%%%%%%%%%%%%%%%%%%%%%%
%%                                                                 %%
%% Please do not use \input{...} to include other tex files.       %%
%% Submit your LaTeX manuscript as one .tex document.              %%
%%                                                                 %%
%% All additional figures and files should be attached             %%
%% separately and not embedded in the \TeX\ document itself.       %%
%%                                                                 %%
%%%%%%%%%%%%%%%%%%%%%%%%%%%%%%%%%%%%%%%%%%%%%%%%%%%%%%%%%%%%%%%%%%%%%

\documentclass[sn-basic,Numbered,pdflatex]{sn-jnl}

%%%% Standard Packages
%%<additional latex packages if required can be included here>

\usepackage{graphicx}%
\usepackage{multirow}%
\usepackage{amsmath,amssymb,amsfonts}%
\usepackage{amsthm}%
\usepackage{mathrsfs}%
\usepackage[title]{appendix}%
\usepackage{xcolor}%
\usepackage{textcomp}%
\usepackage{manyfoot}%
\usepackage{booktabs}%
\usepackage{algorithm}%
\usepackage{algorithmicx}%
\usepackage{algpseudocode}%
\usepackage{listings}%
%%%%

%%%%%=============================================================================%%%%
%%%%  Remarks: This template is provided to aid authors with the preparation
%%%%  of original research articles intended for submission to journals published
%%%%  by Springer Nature. The guidance has been prepared in partnership with
%%%%  production teams to conform to Springer Nature technical requirements.
%%%%  Editorial and presentation requirements differ among journal portfolios and
%%%%  research disciplines. You may find sections in this template are irrelevant
%%%%  to your work and are empowered to omit any such section if allowed by the
%%%%  journal you intend to submit to. The submission guidelines and policies
%%%%  of the journal take precedence. A detailed User Manual is available in the
%%%%  template package for technical guidance.
%%%%%=============================================================================%%%%

%% Per the spinger doc, new theorem styles can be included using built in style, 
%% but it seems the don't work so commented below
%\theoremstyle{thmstyleone}%
\newtheorem{theorem}{Theorem}%  meant for continuous numbers
%%\newtheorem{theorem}{Theorem}[section]% meant for sectionwise numbers
%% optional argument [theorem] produces theorem numbering sequence instead of independent numbers for Proposition
\newtheorem{proposition}[theorem]{Proposition}%
%%\newtheorem{proposition}{Proposition}% to get separate numbers for theorem and proposition etc.

%% \theoremstyle{thmstyletwo}%
\theoremstyle{remark}
\newtheorem{example}{Example}%
\newtheorem{remark}{Remark}%

%% \theoremstyle{thmstylethree}%
\theoremstyle{definition}
\newtheorem{definition}{Definition}%

\usepackage{booktabs}
\usepackage{longtable}
\usepackage{array}
\usepackage{multirow}
\usepackage{wrapfig}
\usepackage{float}
\usepackage{colortbl}
\usepackage{pdflscape}
\usepackage{tabu}
\usepackage{threeparttable}
\usepackage{threeparttablex}
\usepackage[normalem]{ulem}
\usepackage{makecell}
\usepackage{xcolor}


\raggedbottom




% tightlist command for lists without linebreak
\providecommand{\tightlist}{%
  \setlength{\itemsep}{0pt}\setlength{\parskip}{0pt}}





\begin{document}


\title[Article Title runing]{Socioeconomic disparities in child
malnutrition: An analysis of evidence from Kenya Demographic and Health
Survey, 2014 - 2022}

%%=============================================================%%
%% Prefix	-> \pfx{Dr}
%% GivenName	-> \fnm{Joergen W.}
%% Particle	-> \spfx{van der} -> surname prefix
%% FamilyName	-> \sur{Ploeg}
%% Suffix	-> \sfx{IV}
%% NatureName	-> \tanm{Poet Laureate} -> Title after name
%% Degrees	-> \dgr{MSc, PhD}
%% \author*[1,2]{\pfx{Dr} \fnm{Joergen W.} \spfx{van der} \sur{Ploeg} \sfx{IV} \tanm{Poet Laureate}
%%                 \dgr{MSc, PhD}}\email{iauthor@gmail.com}
%%=============================================================%%

\author*[1,2]{\fnm{Amos} \sur{Okutse} }\email{\href{mailto:amos_okutse@brown.edu}{\nolinkurl{amos\_okutse@brown.edu}}}

\author[2]{\fnm{Second} \sur{Author} }

\author[2]{\fnm{Third} \sur{Author} }

\author[2]{\fnm{Fourth} \sur{Author} }



  \affil*[1]{\orgdiv{Department of Biostatistics}, \orgname{Brown
University School of Public
Health}, \orgaddress{\city{Providence}, \country{USA}, \postcode{2903}, \state{RI}, \street{121
South Main St}}}
  \affil[2]{\orgname{Other Organisation}}
  \affil[3]{\orgname{Other Organisation}}

\abstract{\textbf{Purpose}: The abstract serves both as a general
introduction to the topic and as a brief, non-technical summary of the
main results and their implications. The abstract must not include
subheadings (unless expressly permitted in the journal's Instructions to
Authors), equations or citations. As a guide the abstract should not
exceed 200 words. Most journals do not set a hard limit however authors
are advised to check the author instructions for the journal they are
submitting to.

\textbf{Methods:} The abstract serves both as a general introduction to
the topic and as a brief, non-technical summary of the main results and
their implications. The abstract must not include subheadings (unless
expressly permitted in the journal's Instructions to Authors), equations
or citations. As a guide the abstract should not exceed 200 words. Most
journals do not set a hard limit however authors are advised to check
the author instructions for the journal they are submitting to.

\textbf{Results:} The abstract serves both as a general introduction to
the topic and as a brief, non-technical summary of the main results and
their implications. The abstract must not include subheadings (unless
expressly permitted in the journal's Instructions to Authors), equations
or citations. As a guide the abstract should not exceed 200 words. Most
journals do not set a hard limit however authors are advised to check
the author instructions for the journal they are submitting to.

\textbf{Conclusion:} The abstract serves both as a general introduction
to the topic and as a brief, non-technical summary of the main results
and their implications. The abstract must not include subheadings
(unless expressly permitted in the journal's Instructions to Authors),
equations or citations. As a guide the abstract should not exceed 200
words. Most journals do not set a hard limit however authors are advised
to check the author instructions for the journal they are submitting
to.\}}

\keywords{Child malnutrition, Decomposition, Socioeconomic
disparities, Kenya, Stunting, Underweight, Wasting, Demographic Health
Survey}



\maketitle

\hypertarget{sec1}{%
\section{Introduction}\label{sec1}}

Child Malnutrition remains a dominant public health challenge globally.
In 2022, for instance, about 148.1 million (22.3\%) children below 5
years were stunted, whereas 45 million (6.8\%) and 37 million (5.6\%)
were wasted and overweight, respectively \citep{who2023}. While there
has been some progress in the actualization of the global nutrition
targets, this progress is slow, and the levels of malnutrition continue
to persist. Africa and Asia account for almost half the world's child
malnutrition burden \citep{who2023}.

In the East African region, stunting prevalence (32.6\%) was higher than
the global average (22.3\%), whereas wasting and overweight were 5.2\%
and 4.0\%, respectively \citep{IEG2022}. With the possibility of
suffering from more than one form of malnutrition, children remain
largely susceptible to the perilous effects posed by this condition.
According to nutrition statistics, 3.62\% of all children under the age
of five years (15.95 million) have been reported as being both stunted
and wasted, whereas 1.87\% of all children (8.23 million) have been
reported to experience both stunting and overweight globally
\citep{global}.

In the first half of 2022, Kenya reported about 942,000 cases of acute
malnutrition among children between 6 and 59 months
\citep{Bhavnani2023}. According to the 2022 Kenya Demographic and Health
Survey (KDHS) \citep{KNBSICF2023}, 18\% of children under 5 are stunted
(chronically undernourished), 5\% are wasted (acutely malnourished),
whereas 3\% and 10\% are overweight and underweight, respectively. While
Kenya has substantially reduced the burden of child malnutrition,
undernutrition is estimated to cost the country over US\$38.3 billion in
Gross Domestic Product (GDP) following losses in workforce labor and
productivity for 2010--2030 \citep{USAID2018}.

Malnutrition denotes ``a state of nutrition in which a deficiency, or an
excess, of energy, protein, and micronutrient causes measurable adverse
effects on tissue/ body form (body shape, size, and composition),
function, and clinical outcome'' \citep{stratton2003}. Malnutrition has
been attributed to several diverse interlinked factors with detrimental
short and long-term effects \citep{Pelletier1995, Pelletier2003}. Not
only does it affect the physical and cognitive development of a child,
but it also drastically increases their risk of infections and
contributes negatively to their mortality and morbidity
\citep{Rice2000, jonah2018, victora2008, kar2008, mendez1999, walker2015, rabbani2016}.

Stunting, underweight, and wasting remain the recommended three
indicators of malnutrition \citep{jonah2018}. Stunting refers to low
height for age and reflects the growth in linear terms achieved at the
age at which the anthropometric measurements were taken. Underweight is
low weight for age, resulting from a short-term lack of food. In
contrast, wasting is severe undernutrition resulting from inadequate
food intake and infections \citep{jonah2018}. In children under 5 years,
stunting is the most significant measure of overall health and
well-being capable of highlighting salient social disparities
\citep{deOnis2016}. Moreover, because stunting measures linear growth in
children, it is considered an accurate measure of malnutrition in the
long term due to its insensitivity to variations in food consumption
\citep{Hoddinott2013, Zere2003}.

\hypertarget{socioeconomic-disparities-in-child-malnutrition}{%
\subsection{Socioeconomic disparities in child
malnutrition}\label{socioeconomic-disparities-in-child-malnutrition}}

Kenya is classified as a middle-income country based on its Gross
National Income (GNI) per capita. Under this classification frame,
countries are classified into three categories based on their income as
either low, middle, or high-income countries \citep{jonah2018}. A
country's attainment of the middle-income classification status is often
seen as an indication of progress resulting from such activities as
heightened investments across all government sectors and improved
productivity. Shifts in a country's classification from low to middle,
then to high income, indicate economic advancement. As expected of
growth, such advancements are expected to impact the well-being of a
country's population positively. For instance, economic advancements are
expected to create employment opportunities, translating into increased
disposable income, improved health, and education
\citep{Bloom2004, Ranis2000}. Improved living standards following
economic advancement are expected to translate into exceptional and
improved nutritional consequences for children and adults.

The economic status associated with a country has been shown in previous
studies to result in improved health status of a population
\citep{Marmot2002, Ettner1996}. However, economic advancement does not
necessarily translate to equitable distribution of positive prospects
across the population. Often, these tend to be skewed, with some groups
benefiting more than others.

Kenya has made commendable progress in reducing the burden of
malnutrition as part of the Standard Development Goals (SDGs), which
have considerably reduced the stunting rate. Even so, the overall
prevalence of the condition remains larger than those observed for other
forms of malnutrition \citep{jonah2018, Haddad2015}. Given the danger
malnutrition poses to child growth, survival, and well-being, its
consequences are of substantial interest to the government, public
health professionals, and policymakers.

This study contributes significantly to the available knowledge on
socioeconomic disproportions in the Kenyan child malnutrition burden,
examining trends in stunting, underweight, and wasting across
socioeconomic groups, geographical locations, and selected household,
child, maternal, and paternal characteristics. We also examine the
determinants of child malnutrition and employ standard procedures of
inequality to quantify the trends {[}36{]} and decompose vicissitudes in
the concentration indices to determine factors that drive socioeconomic
disparities in child malnutrition in Kenya. The study utilizes current
data from the Kenya Demographic Health Survey (DHS) (2014 to 2022) to
comprehensively analyze the scope of the problem, specifically focusing
on children below five years.

\hypertarget{sec11}{%
\section{Methods}\label{sec11}}

\hypertarget{study-data}{%
\subsection{Study data}\label{study-data}}

We utilized data from the 2014 and 2022 Kenya Demographic and Health
Surveys (KDHS) (standard DHS). These surveys adopted a two-stage
stratified cluster sampling approach: clusters in the first stage and
households in the second. In 2014, a response rate of 99\% was achieved
from 39,679 households, and in 2022, a 98\% response rate from 38,731
occupied households {[}KNBSICF2015; \citet{KNBSICF2023}{]}. Analyses
considered all live children (0-59 months) of interviewed mothers,
excluding those with missing anthropometric data. The data was weighted
for non-response and used with DHS authorization.

\hypertarget{variables}{%
\subsection{Variables}\label{variables}}

\hypertarget{outcome}{%
\subsubsection{Outcome}\label{outcome}}

Malnutrition was characterized by stunting (low height-for-age z-scores,
HAZ), underweight (low weight-for-age z-scores, WAZ), and wasting (low
weight-for-height z-scores, WHZ) \citep{kien_trends_2016, jonah2018}.
Stunting indicates a child's linear growth at a given age. Underweight
results from short-term food deprivation, while wasting stems from both
inadequate food intake and infections \citep{jonah2018}.

In children under five, a HAZ, WAZ, or WHZ between -2 and -3 standard
deviations (SD) below the median suggests moderate stunting,
underweight, or wasting, respectively. Z-scores less than -3 SD below
the World Health Organization's (WHO) child growth standards median
indicate severe conditions \citep{WHO2010}.

We categorized children with HAZ, WAZ, and WHZ scores below -2 SD of the
WHO growth standards median as stunted, underweight, or wasted,
respectively. Notably, stunting in children under five highlights
chronic undernutrition, whereas wasting and underweight can imply both
acute and chronic malnutrition \citep{kien_trends_2016}.

\hypertarget{covariates}{%
\subsubsection{Covariates}\label{covariates}}

In this study, we considered a comprehensive set of determinants linked
to child malnutrition. Child-specific variables included age (in
months), gender, place and region of residence, delivery location, and
birth order. At the household level, we accounted for characteristics
such as religion and socioeconomic status. Maternal indicators included
age, education level, and birth interval, while we also considered
paternal education. These determinants are grounded in existing
literature and were available in our data set
\citep{jonah2018, kien_trends_2016, Farah2019}.

\hypertarget{statistical-analysis}{%
\subsection{Statistical analysis}\label{statistical-analysis}}

\hypertarget{analysis-of-disparities-in-child-malnutrition}{%
\subsubsection{Analysis of disparities in child
malnutrition}\label{analysis-of-disparities-in-child-malnutrition}}

The extent and trends of socioeconomic disparities in stunting,
underweight, and wasting were quantified using concentration indices
(CIs) estimated based on the corresponding z-scores
\citep{odonnell_analyzing_2008, Wagstaff1991, Wagstaff2000}.
Concentration indices quantify the socioeconomic disparities in a health
variable and allow assessment of the extent and levels of disparities.
CIs were computed as double the area between the concentration curve and
the line of equality -- the \(45^{\circ}\) line.

According to O'Donnell et al. \citep{odonnell_analyzing_2008}:
\begin{equation}
CI = \frac{2}{\mu} \textrm{cov}(h, r)
\label{eq:one}
\end{equation}

In Equation \ref{eq:one}, \(\mu\) is the average of malnutrition
(stunting, underweight, and wasting) in children under five children,
\(h\) denotes observation-specific child malnutrition, and \(r\) is the
rank of the socioeconomic status of a household. The CI of a given
health variable usually takes values between -1 and +1, with 0
suggesting perfect equity of the health variable between the poorest and
the richest socioeconomic groups. Negative values suggest a higher
concentration of malnutrition among the poorest group, and positive
values suggest a higher concentration of inequity among the richest
socioeconomic group
\citep{Akombi2019, jonah2018, kien_trends_2016, Wagstaff2000}. As in
Kien et al. \citep{kien_trends_2016}, the continuous forms of the
variables for stunting, underweight, and wasting were employed to
enhance precision.

\hypertarget{analysis-of-determinants-of-child-malnutrition}{%
\subsubsection{Analysis of determinants of child
malnutrition}\label{analysis-of-determinants-of-child-malnutrition}}

Determinants of child malnutrition were investigated using binary
logistic regression. Separate models were fitted for stunting,
underweight, and wasting. Odds ratios were computed for each adjustment
covariate to examine associations between the child malnutrition
indicators and each explanatory variable in \ref{covariates}. Results
from the fitted logistic regression models were used in the construction
and decomposition of the Wagstaff normalized CIs described in
\ref{decompose}.

\hypertarget{decompose}{%
\subsubsection{Decomposition of socioeconomic inequities in child
malnutrition}\label{decompose}}

Contributions of the determinants of malnutrition in children under five
to the observed socioeconomic disparities were examined through a
decomposition analysis. These decompositions were restricted to stunting
and underweight, indicators that exhibited substantial differences
between 2014 and 2022. This analysis framework utilized categorical
forms of the response variables.

We considered a linear regression model where the response variable
(\(y\)) is modeled as a linear combination of the \(k\) determinants
(\(X_k\)) as: \begin{equation}
y = \alpha + \sum_{k} \beta_k X_k + \epsilon
\label{eq:two}
\end{equation} where \(\beta_k\) denotes the coefficient of \(X_k\) (the
set of explanatory variables) and \(\epsilon\) is the error term.

In terms of the CI for the response \(y\), \ref{eq:two} reduces to:
\begin{equation}
CI = \sum_{k} \left( \frac{\beta_k \bar{X}_k}{\mu} \right) CI_k + \frac{G CI_{\epsilon}}{\mu}
\label{eq:three}
\end{equation} where \(\mu\) denotes the average of \(y\), \(\bar{X}_k\)
denotes the mean of the \(k^{th}\) variable, \(\beta_k\) denotes the
coefficient of each determinant, \(CI_k\) denotes the CI of each of the
regressors in the model, and \(G CI_\epsilon\) denotes the generalized
concentration index for the error term, \(\epsilon\).

Equation \ref{eq:three} has two components: the explained
\(\left( (\beta_k \bar{X}_k)/ \mu)CI_k\right)\) and the unexplained
component (\(GCI_{\epsilon}/ \mu\)). \((\beta_k \bar{X}_k)/\mu\) is the
elasticity --the effect of each \$CI\_k \$ on the overall CI of the
outcome variable, \(y\) \citep{Wagstaff2000, Wagstaff2003}.

We employed the Wagstaff normalization technique for CI values and used
total differential decomposition to decipher the determinants'
contribution to CI variations. This decomposition follows from Wagstaff
et al. \citep{Wagstaff2003}. It permits approximation of the effects of
child malnutrition disparities on variations in regression coefficients,
variations in means of malnutrition determinants, and variations in the
extent of inequity in child malnutrition determinants. These
decompositions were applied to height-for-age and weight-for-age
z-scores.

The formula for the decomposition applied was: \begin{equation}
\begin{split}
dC & = \text{-}\frac{c}{\mu}d\alpha + \sum_{k}\frac{\bar{X}_k}{\mu} \left( CI_k \text{–} CI\right)d\beta_k \\
& + \sum_{k} \frac{\beta_k}{\mu} \left( CI_k \text{-} CI\right)d \bar{X}_k + \sum_{k} \frac{\beta_k \bar{X}_k}{\mu} dCI_k + d\frac{GCI}{\mu}{\epsilon}
\end{split}
\label{eq:four}
\end{equation} where \(dC\) denotes the overall change in the CI,
\(d\alpha\) constant value, \(d\beta_k\) the coefficients of the
determinants, \(d\bar{X}_k\) mean values of the determinants, \(dCI_k\)
determinant-specific CI and \(d((GCI)/\mu)_{\epsilon}\), the error term
\citep{Wagstaff2003}.

\hypertarget{sec2}{%
\section{Results}\label{sec2}}

\hypertarget{weighted-prevalence-of-child-malnutrition}{%
\subsection{Weighted prevalence of child
malnutrition}\label{weighted-prevalence-of-child-malnutrition}}

Figure \ref{tab:one} highlights the weighted prevalence of under five
child malnutrition by selected child, household, maternal, and paternal
characteristics grouped by the survey year.

\renewcommand{\arraystretch}{0.8}

\begin{landscape}\begingroup\fontsize{7}{9}\selectfont

\begin{longtable}[t]{>{\raggedright\arraybackslash}p{1.5cm}>{\centering\arraybackslash}p{0.3cm}>{\centering\arraybackslash}p{1.5cm}>{\centering\arraybackslash}p{0.2cm}>{\centering\arraybackslash}p{0.3cm}>{\centering\arraybackslash}p{1.5cm}>{\centering\arraybackslash}p{0.2cm}>{\centering\arraybackslash}p{0.3cm}>{\centering\arraybackslash}p{1.5cm}>{\centering\arraybackslash}p{0.2cm}>{\centering\arraybackslash}p{0.3cm}>{\centering\arraybackslash}p{1.5cm}>{\centering\arraybackslash}p{0.2cm}>{\centering\arraybackslash}p{0.3cm}>{\centering\arraybackslash}p{1.5cm}>{\centering\arraybackslash}p{0.2cm}>{\centering\arraybackslash}p{0.3cm}>{\centering\arraybackslash}p{1.5cm}>{\centering\arraybackslash}p{0.2cm}}
\caption{\label{tab:one}Weighted prevalence of stunting, underweight, and wasting among children under five years by selected child, household, maternal, and paternal characteristics (2014 -- 2022)}\\
\toprule
\multicolumn{1}{c}{\textbf{ }} & \multicolumn{9}{c}{\textbf{2014}} & \multicolumn{9}{c}{\textbf{2022}} \\
\cmidrule(l{3pt}r{3pt}){2-10} \cmidrule(l{3pt}r{3pt}){11-19}
\multicolumn{1}{c}{\textbf{ }} & \multicolumn{3}{c}{\textbf{Stunted}} & \multicolumn{3}{c}{\textbf{Underweight}} & \multicolumn{3}{c}{\textbf{Wasted}} & \multicolumn{3}{c}{\textbf{Stunted)}} & \multicolumn{3}{c}{\textbf{Underweight}} & \multicolumn{3}{c}{\textbf{Wasted}} \\
\multicolumn{1}{c}{ } & \multicolumn{3}{c}{(HAZ$<$-2SD)} & \multicolumn{3}{c}{(WAZ$<$-2SD)} & \multicolumn{3}{c}{(WHZ$<$-2SD)} & \multicolumn{3}{c}{(HAZ$<$-2SD)} & \multicolumn{3}{c}{(WAZ$<$-2SD)} & \multicolumn{3}{c}{(WHZ$<$-2SD)} \\
\cmidrule(l{0pt}r{0pt}){2-4} \cmidrule(l{0pt}r{0pt}){5-7} \cmidrule(l{0pt}r{0pt}){8-10} \cmidrule(l{0pt}r{0pt}){11-13} \cmidrule(l{0pt}r{0pt}){14-16} \cmidrule(l{0pt}r{0pt}){17-19}
\textbf{ } & \textbf{$n$} & \textbf{$n(\%)$} & \textbf{$P$} & \textbf{$n$} & \textbf{$n(\%)$} & \textbf{$P$} & \textbf{$n$} & \textbf{$n(\%)$} & \textbf{$P$} & \textbf{$n$} & \textbf{$n(\%)$} & \textbf{$P$} & \textbf{$n$} & \textbf{$n(\%)$} & \textbf{$P$} & \textbf{$n$} & \textbf{$n(\%)$} & \textbf{$P$}\\
\midrule
\endfirsthead
\caption[]{(continued)}\\
\toprule
\textbf{ } & \textbf{$n$} & \textbf{$n(\%)$} & \textbf{$P$} & \textbf{$n$} & \textbf{$n(\%)$} & \textbf{$P$} & \textbf{$n$} & \textbf{$n(\%)$} & \textbf{$P$} & \textbf{$n$} & \textbf{$n(\%)$} & \textbf{$P$} & \textbf{$n$} & \textbf{$n(\%)$} & \textbf{$P$} & \textbf{$n$} & \textbf{$n(\%)$} & \textbf{$P$}\\
\midrule
\endhead

\endfoot
\bottomrule
\multicolumn{19}{l}{\rule{0pt}{1em}\textit{Note: } HAZ: height-for-age z-score; WAZ: weight-for-age z-score; WHZ: weight-for-height z-score; SD: standard deviation;}\\
\multicolumn{19}{l}{\rule{0pt}{1em}\textsuperscript{*} $P$ based on a Pearson chi-square test for categorical variable and T-test for continuous variables.}\\
\endlastfoot
Weighted n & 17291 & 4466 (25.8) &  & 17291 & 1841 (10.6) &  & 17291 & 701 (4.1) &  & 15336 & 2665 (17.5) &  & 15368 & 1543 (10.0) &  & 15329 & 752 (4.9) & \\
Child age, mean (SE) &  & 30.8 (0.2) & 0.00 &  & 31.8 (0.5) & 0.00 &  & 26.1 (1.0) & 0.00 &  & 27.7 (0.4) & 0.08 &  & 30.8 (0.5) & 0.00 &  &  & \\
Birth interval, mean (SE) &  & 39.5 (0.5) & 0.00 &  & 38.2 (0.7) & 0.00 &  & 39.8 (1.2) & 0.00 &  & 43.8 (0.8) & 0.00 &  & 41.8 (1.0) & 0.00 &  &  & \\
Birth order , mean (SE) &  & 3.6 (0.1) & 0.00 &  & 3.8 (0.1) & 0.00 &  & 3.6 (0.1) & 0.01 &  & 3.3 (0.1) & 0.00 &  & 3.5 (0.1) & 0.00 &  &  & \\
\textbf{Child sex} & \textbf{} & \textbf{} & \textbf{0.00} & \textbf{} & \textbf{} & \textbf{0.00} & \textbf{} & \textbf{} & \textbf{0.09} & \textbf{} & \textbf{} & \textbf{0.00} & \textbf{} & \textbf{} & \textbf{} & \textbf{} & \textbf{} & \textbf{0.01}\\
\addlinespace
\hspace{1em}Male & 8763 & 2586 (57.9) &  & 8763 & 1028 (55.9) &  & 8763 & 382 (54.6) &  & 7755 & 1523 (57.2) &  & 7767 & 857 (55.6) &  & 7758 & 420 (55.9) & \\
\hspace{1em}Female & 8528 & 1880 (42.1) &  & 8528 & 812 (44.1) &  & 8528 & 318 45.4) &  & 7581 & 1142 (42.8) &  & 7601 & 685 (44.4) &  & 7571 & 331 (44.1) & \\
\textbf{Delivery place} & \textbf{} & \textbf{} & \textbf{0.00} & \textbf{} & \textbf{} & \textbf{0.00} & \textbf{} & \textbf{} & \textbf{0.00} & \textbf{} & \textbf{} & \textbf{0.00} & \textbf{} & \textbf{} & \textbf{0.00} & \textbf{} & \textbf{} & \textbf{0.00}\\
\hspace{1em}Home & 6513 & 2157 (48.5) &  & 6512 & 1033 (56.4) &  & 6512 & 390 (55.9) &  & 1110 & 302 (16.9) &  & 1116 & 205 (23.4) &  & 1116 & 104 (25.7) & \\
\hspace{1em}Public & 7919 & 1749 (39.3) &  & 7919 & 629 (34.3) &  & 7919 & 235 (33.6) &  & 6095 & 1147 (64.2) &  & 6114 & 513 (58.6) &  & 6094 & 217 (53.1) & \\
\addlinespace
\hspace{1em}Private & 2637 & 495 (11.1) &  & 2637 & 153 (8.4) &  & 2637 & 68 (9.7) &  & 2217 & 321 (18.0) &  & 2217 & 150 (17.2) &  & 2202 & 80 (53.1) & \\
\hspace{1em}Other & 176 & 50 (1.1) &  & 176 & 16 (0.9) &  & 176 & 5 (0.7) &  & 50 & 14 (0.8) &  & 50 & 7 (0.8) &  & 49 & 5 (1.4) & \\
\textbf{Residence} & \textbf{} & \textbf{} & \textbf{0.00} & \textbf{} & \textbf{} & \textbf{0.00} & \textbf{} & \textbf{} & \textbf{0.03} & \textbf{} & \textbf{} & \textbf{0.00} & \textbf{} & \textbf{} & \textbf{0.00} & \textbf{} & \textbf{} & \textbf{0.01}\\
\hspace{1em}Urban & 5927 & 1168 (26.1) &  & 5926 & 397 (21.6) &  & 5926 & 200 (28.6) &  & 5411 & 662 (24.8) &  & 5424 & 361 (23.4) &  & 5410 & 215 (28.7) & \\
\hspace{1em}Rural & 11364 & 3298 (73.9) &  & 11364 & 1443 (78.4) &  & 11364 & 500 (71.4) &  & 9924 & 2003 (75.2) &  & 9944 & 1182 (76.6) &  & 9918 & 536 (71.3) & \\
\addlinespace
\textbf{Religion} & \textbf{} & \textbf{} & \textbf{0.00} & \textbf{} & \textbf{} & \textbf{0.00} & \textbf{} & \textbf{} & \textbf{0.00} & \textbf{} & \textbf{} & \textbf{0.01} & \textbf{} & \textbf{} & \textbf{0.03} & \textbf{} & \textbf{} & \textbf{0.00}\\
\hspace{1em}Catholic & 3097 & 728 (16.3) &  & 3097 & 296 (16.1) &  & 3097 & 126 (18.1) &  & 2670 & 445 (16.7) &  & 2676 & 276 (17.9) &  & 2667 & 128 (17.1) & \\
\hspace{1em}Protestant & 12228 & 3200 (71.8) &  & 12228 & 1257 (68.4) &  & 12228 & 444 (63.5) &  & 10527 & 1864 (70.0) &  & 10541 & 1012 (65.6) &  & 10513 & 432 (57.5) & \\
\hspace{1em}Muslim & 1440 & 346 (7.8) &  & 1440 & 195 (10.6) &  & 1440 & 106 (15.2) &  & 1485 & 232 (8.7) &  & 1489 & 190 (12.3) &  & 1494 & 159 (21.2) & \\
\hspace{1em}Atheist & 457 & 181 (4.1) &  & 457 & 79 (4.3) &  & 457 & 18 (2.6) &  & 213 & 59 (2.2) &  & 217 & 24 (1.6) &  & 213 & 12 (1.7) & \\
\addlinespace
\hspace{1em}Other & 42 & 5.4 (0.1) &  & 42 & 8.5 (0.5) &  & 42 & 4 (0.7) &  & 439 & 63 (2.4) &  & 442 & 40 (2.6) &  & 440 & 19 (2.5) & \\
\textbf{Economic status} & \textbf{} & \textbf{} & \textbf{0.00} & \textbf{} & \textbf{} & \textbf{0.00} & \textbf{} & \textbf{} & \textbf{0.00} & \textbf{} & \textbf{} & \textbf{0.00} & \textbf{} & \textbf{} & \textbf{0.00} & \textbf{} & \textbf{} & \textbf{0.00}\\
\hspace{1em}Poorest & 4178 & 1489 (33.4) &  & 4178 & 792 (43.0) &  & 4178 & 303 (43.2) &  & 3583 & 986 (37.0) &  & 3596 & 679 (44.0) &  & 3583 & 340 (45.2) & \\
\hspace{1em}Poorer & 3631 & 1099 (24.6) &  & 3631 & 435 (23.7) &  & 3631 & 116 (16.6) &  & 2840 & 598 (22.4) &  & 2846 & 286 (18.5) &  & 2849 & 88 (11.8) & \\
\hspace{1em}Middle & 3182 & 808 (18.1) &  & 3182 & 286 (15.6) &  & 3182 & 117 (16.8) &  & 2703 & 439 (16.5) &  & 2707 & 249 (16.1) &  & 2701 & 116 (15.5) & \\
\addlinespace
\hspace{1em}Richer & 2969 & 620 (13.9) &  & 2969 & 204 (11.1) &  & 2969 & 76 (11.0) &  & 3052 & 354 (13.3) &  & 3062 & 190 (12.4) &  & 3040 & 124 (16.5) & \\
\hspace{1em}Richest & 3330 & 446 (10.0) &  & 3330 & 122 (6.7) &  & 3330 & 86 (12.4) &  & 3155 & 286 (10.7) &  & 3154 & 137 (8.9) &  & 3154 & 82 (10.9) & \\
\textbf{Mothers education} & \textbf{} & \textbf{} & \textbf{0.00} & \textbf{} & \textbf{} & \textbf{0.00} & \textbf{} & \textbf{} & \textbf{0.00} & \textbf{} & \textbf{} & \textbf{0.00} & \textbf{} & \textbf{} & \textbf{0.00} & \textbf{} & \textbf{} & \textbf{0.00}\\
\hspace{1em}None & 2057 & 628 (14.1) &  & 2057 & 421 (22.9) &  & 2057 & 210 (30.0) &  & 1606 & 354 (13.3) &  & 1614 & 355 (23.0) &  & 1611 & 248 (33.1) & \\
\hspace{1em}Primary & 9735 & 2890 (64.7) &  & 9735 & 1111.7 (60.4) &  & 9735 & 329 (47.0) &  & 5820 & 1283 (48.2) &  & 5834 & 688 (44.6) &  & 5829 & 257 (34.2) & \\
\addlinespace
\hspace{1em}Higher & 5497 & 946 (21.2) &  & 5497 & 307 (16.7) &  & 5497 & 161.7 (23.1) &  & 7909 & 1027 (38.6) &  & 7919 & 500 (32.4) &  & 7888 & 246 (32.7) & \\
\textbf{Mothers age} & \textbf{} & \textbf{} & \textbf{0.20} & \textbf{} & \textbf{} & \textbf{0.01} & \textbf{} & \textbf{} & \textbf{0.35} & \textbf{} & \textbf{} & \textbf{0.00} & \textbf{} & \textbf{} & \textbf{0.54} & \textbf{} & \textbf{} & \textbf{0.11}\\
\hspace{1em}under 24 & 5000 & 1349 (30.2) &  & 5000 & 474 (25.8) &  & 5000 & 182 (26.1) &  & 4084 & 817 (30.7) &  & 4094 & 390 (25.3) &  & 4083 & 173 (23.1) & \\
\hspace{1em}25 - 34 & 8855 & 2228 (49.9) &  & 8855 & 946 (51.4) &  & 8855 & 375 (53.6) &  & 7852 & 1313 (49.3) &  & 7874 & 800 (51.8) &  & 7855 & 394 (52.4) & \\
\hspace{1em}35+ & 3435 & 888 (19.9) &  & 3435 & 420 (22.8) &  & 3435 & 142 (20.3) &  & 3400 & 534 (20.1) &  & 3400 & 353 (22.9) &  & 3390 & 184 (24.5) & \\
\addlinespace
\textbf{Mother employed} & \textbf{} & \textbf{} & \textbf{0.54} & \textbf{} & \textbf{} & \textbf{0.32} & \textbf{} & \textbf{} & \textbf{0.00} & \textbf{} & \textbf{} & \textbf{0.20} & \textbf{} & \textbf{} & \textbf{0.00} & \textbf{} & \textbf{} & \textbf{0.00}\\
\hspace{1em}No & 3041 & 770 (35.9) &  & 3041 & 342 (38.6) &  & 3041 & 149 (49.0) &  & 7596 & 1362 (51.1) &  & 7625 & 853 (55.3) &  & 7602 & 434 (57.8) & \\
\hspace{1em}Yes & 5263 & 1378 (64.1) &  & 5263 & 545 (61.4) &  & 5263 & 155 (51.0) &  & 7739 & 1302 (48.9) &  & 7742 & 689 (44.7) &  & 7726 & 317 (42.2) & \\
\textbf{Fathers education} & \textbf{} & \textbf{} & \textbf{0.00} & \textbf{} & \textbf{} & \textbf{0.00} & \textbf{} & \textbf{} & \textbf{0.00} & \textbf{} & \textbf{} & \textbf{0.00} & \textbf{} & \textbf{} & \textbf{0.00} & \textbf{} & \textbf{} & \textbf{0.00}\\
\hspace{1em}None & 727 & 213 (10.7) &  & 727 & 155 (18.5) &  & 727 & 81 (27.6) &  & 1257 & 304 (14.5) &  & 1263 & 292 (23.8) &  & 1260 & 188 (30.4) & \\
\addlinespace
\hspace{1em}Primary & 3964 & 1183 (59.3) &  & 3964 & 453 (53.9) &  & 3964 & 120 (40.8) &  & 4637 & 961 (45.8) &  & 4651 & 517 (42.0) &  & 4647 & 217 (35.1) & \\
\hspace{1em}Higher & 3019 & 599 (30.0) &  & 3019 & 232 (27.6) &  & 3019 & 93 (31.6) &  & 6629 & 833 (39.7) &  & 6638 & 419 (34.1) &  & 6615 & 213 (34.5) & \\
\textbf{Region} & \textbf{} & \textbf{} & \textbf{0.00} & \textbf{} & \textbf{} & \textbf{0.00} & \textbf{} & \textbf{} & \textbf{} & \textbf{} & \textbf{} & \textbf{0.00} & \textbf{} & \textbf{} & \textbf{0.00} & \textbf{} & \textbf{} & \textbf{0.00}\\
\hspace{1em}Coast & 1774 & 532 (11.9) &  & 1774 & 226 (12.3) &  & 1774 & 75 (10.8) &  & 1441 & 204 (13.2) &  & 1441 & 204 (13.2) &  & 1430 & 91 (12.2) & \\
\hspace{1em}N.Eastern & 557 & 134 (3.0) &  & 557 & 100 (5.5) &  & 557 & 72 (10.3) &  & 562 & 104 (6.8) &  & 562 & 104 (6.8) &  & 562 & 99 (13.2) & \\
\addlinespace
\hspace{1em}Eastern & 2147 & 640 (14.3) &  & 2147 & 259 (14.1) &  & 2147 & 97 (13.9) &  & 1857 & 206 (13.4) &  & 1857 & 206 (13.4) &  & 1850 & 108 (14.4) & \\
\hspace{1em}Central & 1605 & 289 (6.5) &  & 1605 & 78 (4.3) &  & 1605 & 33 (4.7) &  & 1737 & 94 (6.1) &  & 1737 & 94 (6.1) &  & 1730 & 46 (6.2) & \\
\hspace{1em}R. Valley & 5047 & 1502 (33.7) &  & 5047 & 772 (41.9) &  & 5047 & 289 (41.3) &  & 4768 & 623 (40.4) &  & 4768 & 623 (40.4) &  & 4763 & 287 (38.2) & \\
\hspace{1em}Western & 2031 & 506 (11.3) &  & 2031 & 164 (8.9) &  & 2031 & 41 (6.0) &  & 1514 & 117 (7.6) &  & 1514 & 117 (7.6) &  & 1508 & 31 (4.1) & \\
\hspace{1em}Nyanza & 2448 & 556 (12.5) &  & 2448 & 183 (10.0) &  & 2448 & 47 (6.8) &  & 1877 & 106 (6.9) &  & 1877 & 106 (6.9) &  & 1874 & 44 (5.9) & \\
\addlinespace
\hspace{1em}Nairobi & 1678 & 302 (6.8) &  & 1678 & 56 (3.0) &  & 1678 & 43 (6.1) &  & 1610 & 86 (5.6) &  & 1610 & 86 (5.6) &  & 1609 & 43 (5.8) & \\*
\end{longtable}
\endgroup{}
\end{landscape}
\renewcommand{\arraystretch}{1}

\hypertarget{trends-in-child-malnutrition-and-socioeconomic-inequality}{%
\subsection{Trends in child malnutrition and socioeconomic
inequality}\label{trends-in-child-malnutrition-and-socioeconomic-inequality}}

Figure \ref{tab:two} summarizes the prevalence of child malnutrition by
the household socioeconomic status between 2014 and 2022.

\begin{table}[!h]

\caption{\label{tab:two}Malnutrition prevalence by household socioeconomic status, \% (SE)}
\centering
\begin{tabular}[t]{lllllll}
\toprule
\textbf{ } & \textbf{Poorest} & \textbf{Poorer} & \textbf{Middle} & \textbf{Richer} & \textbf{Richest} & \textbf{All}\\
\midrule
\addlinespace[0.3em]
\multicolumn{7}{l}{\textbf{Stunting (height for age $<$ -2 SD)}}\\
\hspace{1em}2014 & 34.2 (0.6) & 30.2 (0.7) & 24.9 (0.8) & 20.6 (0.7) & 12.9 (0.7) & 27.1 (0.3)\\
\hspace{1em}2022 & 25.6 (0.6) & 20.5 (0.7) & 15.4 (0.7) & 11.7 (0.6) & 07.7 (0.6) & 18.0 (0.3)\\
\hspace{1em}Diff-1 & 08.6 (0.8)* & 09.8 (1.0)* & 09.4 (1.0)* & 08.9 (1.0)* & 05.2 (0.9)* & 09.1 (0.4)*\\
\addlinespace[0.3em]
\multicolumn{7}{l}{\textbf{Underweight (weight for age $<$ -2 SD)}}\\
\hspace{1em}2014 & 21.2 (0.5) & 12.7 (0.5) & 09.3 (0.5) & 07.4 (0.5) & 04.1 (0.4) & 13.2 (0.2)\\
\hspace{1em}2022 & 21.8 (0.5) & 10.6 (0.5) & 09.6 (0.5) & 06.2 (0.4) & 04.5 (0.4) & 12.6 (0.3)\\
\hspace{1em}Diff-2 & -00.6 (0.7) & 02.0 (0.8)* & -00.3 (0.7) & 01.2 (0.7) & -00.3 (0.6) & 00.6 (0.3)\\
\addlinespace[0.3em]
\multicolumn{7}{l}{\textbf{Wasting (weight for height $<$ -2 SD)}}\\
\hspace{1em}2014 & 09.4 (0.4) & 03.6 (0.3) & 03.8 (0.3) & 03.2 (0.3) & 02.9 (0.3) & 05.5 (0.2)\\
\hspace{1em}2022 & 12.9 (0.4) & 04.2 (0.4) & 05.3 (0.4) & 04.3 (0.4) & 02.9 (0.3) & 07.2 (0.2)\\
\hspace{1em}Diff-3 & -03.5 (0.6)* & -00.6 (0.4) & -01.6 (5.3)* & -01.1 (0.5) & 00.0 (0.5) & -01.7 (0.3)*\\
\bottomrule
\multicolumn{7}{l}{\rule{0pt}{1em}\textit{Note: }}\\
\multicolumn{7}{l}{\rule{0pt}{1em}Diff-1, Diff-2, Diff-3: difference in under five stunting, underweight, and wasting, respectively.}\\
\multicolumn{7}{l}{\rule{0pt}{1em}SE: standard error; SD: standard deviation}\\
\multicolumn{7}{l}{\rule{0pt}{1em}\textsuperscript{*} significance based on two-sample comparisons of differences in proportions}\\
\end{tabular}
\end{table}

Table \ref{tab:three} presents the concentration indices of under five
child malnutrition.

\begin{table}[!h]

\caption{\label{tab:three}Under five child malnutrition concentration indices (CI), 2014 -- 2022}
\centering
\begin{tabular}[t]{lcccccc}
\toprule
\multicolumn{1}{c}{\textbf{ }} & \multicolumn{2}{c}{\textbf{Stunted (HAZ $<$ -2 SD)}} & \multicolumn{2}{c}{\textbf{Underweight (WAZ $<$ -2 SD)}} & \multicolumn{2}{c}{\textbf{Wasted (WHZ $<$ -2 SD)}} \\
\cmidrule(l{3pt}r{3pt}){2-3} \cmidrule(l{3pt}r{3pt}){4-5} \cmidrule(l{3pt}r{3pt}){6-7}
  & CI (SE) & $P*$ & CI (SE) & $P*$ & CI (SE) & $P*$\\
\midrule
Year 2014 & -0.15 (0.01) & 0.00 & -0.27 (0.02) & 0.00 & 12.37 (22.61) & 0.58\\
Year 2022 & -0.79 (0.01) & 0.00 & -0.88 (0.01) & 0.00 & -1.96 (0.05) & 0.00\\
Diff & -0.64 (0.01) & 0.00 & -0.61 (0.02) & 0.00 & -14.33 (22.62) & 0.53\\
\bottomrule
\multicolumn{7}{l}{\rule{0pt}{1em}\textit{Note: }}\\
\multicolumn{7}{l}{\rule{0pt}{1em}Diff: difference in child malnutrition concentration indices between 2014 and 2022;}\\
\multicolumn{7}{l}{\rule{0pt}{1em}SE: standard error; SD: standard deviation; HAZ: height-for-age Z-score; WAZ: weight-for-age Z-score;}\\
\multicolumn{7}{l}{\rule{0pt}{1em}WHZ: weight-for-height Z-score}\\
\multicolumn{7}{l}{\rule{0pt}{1em}\textsuperscript{*} $P$-value based on a two-tailed independence test comparing the differences in the CIs with a test value of 0}\\
\end{tabular}
\end{table}

\hypertarget{determinants-of-child-malnutrition}{%
\subsection{Determinants of child
malnutrition}\label{determinants-of-child-malnutrition}}

Table \ref{tab:four} presents a summary of the determinants of under
five child stunting, underweight, and wasting based on the multivariable
logistic regression model. Results are based on the analysis of the
aggregate 2014 and 2022 KDHS datasets.

\renewcommand{\arraystretch}{0.8}
\begin{table}[!h]

\caption{\label{tab:four}Determinants of under five child malnutrition, KDHS 2014 -- 2022}
\centering
\begin{tabular}[t]{lllllll}
\toprule
\multicolumn{1}{c}{\textbf{ }} & \multicolumn{2}{c}{\textbf{Stunted}} & \multicolumn{2}{c}{\textbf{Underweight}} & \multicolumn{2}{c}{\textbf{Wasted}} \\
\multicolumn{1}{c}{ } & \multicolumn{2}{c}{(HAZ$<$-2SD)} & \multicolumn{2}{c}{(WAZ$<$-2SD)} & \multicolumn{2}{c}{(WHZ$<$-2SD)} \\
\cmidrule(l{0pt}r{0pt}){2-3} \cmidrule(l{0pt}r{0pt}){4-5} \cmidrule(l{0pt}r{0pt}){6-7}
\textbf{ } & \textbf{AOR 95$\%$ CI} & \textbf{$p$} & \textbf{AOR 95$\%$ CI} & \textbf{$p$} & \textbf{AOR 95$\%$ CI} & \textbf{$p$}\\
\midrule
Year 2014 & 1.04 (0.90 - 1.19) & 0.60 & 0.74 (0.63 - 0.88) & 0.00 & 0.71 (0.56 - 0.91) & 0.01\\
Child age (months) & 1.01 (1.01 - 1.02) & 0.00 & 1.01 (1.00 - 1.01) & 0.00 & 0.99 (0.98 - 0.99) & 0.00\\
Birth interval & 0.99 (0.99 - 1.00) & 0.00 & 0.99 (0.99 - 1.00) & 0.00 & 0.93 (0.87 - 1.00) & 0.04\\
Birth order number & 1.05 (1.01 - 1.10) & 0.01 & 0.96 (0.91 - 1.00) & 0.07 & 0.99 (0.98 - 0.99) & 0.00\\
\textbf{Childs sex} & \textbf{} & \textbf{} & \textbf{} & \textbf{} & \textbf{} & \textbf{}\\
\addlinespace
\hspace{1em}Male & 1.45 (1.30 - 1.62) & 0.00 & 1.35 (1.17 - 1.55) & 0.00 & 1.22 (1.00 - 1.48) & 0.05\\
\hspace{1em}Female & ref &  & ref &  & ref & \\
\textbf{Delivery place} & \textbf{} & \textbf{} & \textbf{} & \textbf{} & \textbf{} & \textbf{}\\
\hspace{1em}Home & ref &  & ref &  & ref & \\
\hspace{1em}Public & 0.81 (0.70 - 0.94) & 0.01 & 0.67 (0.56 - 0.81) & 0.00 & 0.71 (0.54 - 0.94) & 0.02\\
\addlinespace
\hspace{1em}Private & 0.85 (0.69 - 1.06) & 0.15 & 0.67 (0.50 - 0.89) & 0.01 & 0.75 (0.50 - 1.14) & 0.18\\
\hspace{1em}Other & 1.18 (0.72 - 1.93) & 0.50 & 1.19 (0.63 - 2.25) & 0.59 & 1.96 (0.79 - 4.87) & 0.15\\
\textbf{Residence} & \textbf{} & \textbf{} & \textbf{} & \textbf{} & \textbf{} & \textbf{}\\
\hspace{1em}Urban & 0.88 (0.74 - 1.03) & 0.11 & 0.79 (0.64 - 0.99) & 0.04 & 1.01 (0.76 - 1.34) & 0.96\\
\hspace{1em}Rural & ref &  & ref &  & ref & \\
\addlinespace
\textbf{Religion} & \textbf{} & \textbf{} & \textbf{} & \textbf{} & \textbf{} & \textbf{}\\
\hspace{1em}Catholic & 0.42 (0.31 - 0.58) & 0.00 & 0.54 (0.38 - 0.78) & 0.00 & 0.39 (0.26 - 0.59) & 0.00\\
\hspace{1em}Protestant & 0.47 (0.35 - 0.63) & 0.00 & 0.53 (0.38 - 0.73) & 0.00 & 0.48 (0.33 - 0.71) & 0.00\\
\hspace{1em}Muslim & 0.35 (0.24 - 0.51) & 0.00 & 0.34 (0.22 - 0.53) & 0.00 & 0.41 (0.24 - 0.69) & 0.00\\
\hspace{1em}Other & 0.47 (0.26 - 0.83) & 0.01 & 0.59 (0.31 - 1.13) & 0.11 & 0.54 (0.24 - 1.20) & 0.13\\
\addlinespace
\hspace{1em}Atheist & ref &  & ref &  & ref & \\
\textbf{Economic status} & \textbf{} & \textbf{} & \textbf{} & \textbf{} & \textbf{} & \textbf{}\\
\hspace{1em}Poorest & 1.67 (1.24 - 2.26) & 0.00 & 1.08 (0.74 - 1.56) & 0.69 & 0.90 (0.58 - 1.42) & 0.66\\
\hspace{1em}Poorer & 1.46 (1.08 - 1.98) & 0.01 & 0.80 (0.56 - 1.16) & 0.24 & 0.51 (0.31 - 0.82) & 0.01\\
\hspace{1em}Middle & 1.32 (0.98 - 1.79) & 0.07 & 0.85 (0.59 - 1.23) & 0.39 & 1.05 (0.65 - 1.69) & 0.85\\
\addlinespace
\hspace{1em}Richer & 1.23 (0.91 - 1.66) & 0.07 & 0.76 (0.53 - 1.11) & 0.16 & 0.85 (0.55 - 1.31) & 0.46\\
\hspace{1em}Richest & ref &  &  &  & ref & \\
\textbf{Mothers education} & \textbf{} & \textbf{} & \textbf{} & \textbf{} & \textbf{} & \textbf{}\\
\hspace{1em}None & ref &  & ref &  & ref \vphantom{1} & \\
\hspace{1em}Primary & 1.12 (0.92 - 1.37) & 0.27 & 0.82 (0.65 - 1.03) & 0.08 & 0.61 (0.45 - 0.83) & 0.00\\
\addlinespace
\hspace{1em}Higher & 0.81 (0.64 - 1.04) & 0.10 & 0.46 (0.34 - 0.63) & 0.00 & 0.42 (0.27 - 0.65) & 0.00\\
\textbf{Mother's age (years)} & \textbf{} & \textbf{} & \textbf{} & \textbf{} & \textbf{} & \textbf{}\\
\hspace{1em}Under 24 & ref &  & ref &  & ref & \\
\hspace{1em}25 - 34 & 0.78 (0.66 - 0.93) & 0.00 & 1.34 (1.08 - 1.66) & 0.01 & 1.25 (0.94 - 1.66) & 0.13\\
\hspace{1em}35+ & 0.65 (0.50 - 0.84) & 0.00 & 1.49 (1.08 - 2.05) & 0.00 & 1.60 (1.04 - 2.46) & 0.03\\
\addlinespace
\textbf{Mother employed} & \textbf{} & \textbf{} & \textbf{} & \textbf{} & \textbf{} & \textbf{}\\
\hspace{1em}No & ref &  & ref &  & ref & \\
\hspace{1em}Yes & 1.06 (0.93 - 1.21) & 0.37 & 1.02 (0.87 - 1.21) & 0.80 & 0.91 (0.72 - 1.15) & 0.44\\
\textbf{Fathers education} & \textbf{} & \textbf{} & \textbf{} & \textbf{} & \textbf{} & \textbf{}\\
\hspace{1em}None & ref &  & ref &  & ref & \\
\addlinespace
\hspace{1em}Primary & 0.91 (0.75 - 1.11) & 0.36 & 0.67 (0.53 - 0.85) & 0.00 & 0.64 (0.46 - 0.89) & 0.01\\
\hspace{1em}Higher & 0.75 (0.59 - 0.94) & 0.01 & 0.65 (0.49 - 0.86) & 0.00 & 0.64 (0.43 - 0.96) & 0.03\\
\textbf{Region} & \textbf{} & \textbf{} & \textbf{} & \textbf{} & \textbf{} & \textbf{}\\
\hspace{1em}Coast & 0.62 (0.41 - 0.93) & 0.02 & 0.85 (0.54 - 1.33) & 0.48 & 0.70 (0.39 - 1.26) & 0.23\\
\hspace{1em}N.eastern & 0.34 (0.22 - 0.55) & 0.00 & 0.82 (0.52 - 1.29) & 0.38 & 1.29 (0.71 - 2.36) & 0.40\\
\addlinespace
\hspace{1em}Eastern & 0.62 (0.42 - 0.91) & 0.01 & 0.87 (0.58 - 1.29) & 0.49 & 0.98 (0.56 - 1.70) & 0.93\\
\hspace{1em}Central & 0.48 (0.31 - 0.74) & 0.00 & 0.58 (0.36 - 0.96) & 0.03 & 0.48 (0.23 - 1.00) & 0.05\\
\hspace{1em}R. Valley & 0.51 (0.35 - 0.74) & 0.00 & 0.79 (0.54 - 1.15) & 0.21 & 0.84 (0.50 - 1.40) & 0.50\\
\hspace{1em}Western & 0.39 (0.26 - 0.59) & 0.00 & 0.48 (0.30 - 0.76) & 0.00 & 0.42 (0.21 - 0.82) & 0.01\\
\hspace{1em}Nyanza & 0.36 (0.24 - 0.54) & 0.00 & 0.52 (0.34 - 0.79) & 0.00 & 0.45 (0.25 - 0.83) & 0.01\\
\addlinespace
\hspace{1em}Nairobi & ref &  & ref &  & ref & \\
\bottomrule
\end{tabular}
\end{table}
\renewcommand{\arraystretch}{1}

\hypertarget{decomposition-of-the-concentration-indices-for-stunting-and-underweight}{%
\subsection{Decomposition of the concentration indices for stunting and
underweight}\label{decomposition-of-the-concentration-indices-for-stunting-and-underweight}}

In Table \ref{tab:five} we present each determinant of child
malnutrition and its percentage contribution to the observed inequality
in child stunting and underweight for the period 2014 and 2022. Negative
values suggest contributions to decreases in socioeconomic inequality
whereas positive values indicate contributions to increase in
inequality.

\begin{sidewaystable}[!h]

\caption{\label{tab:five}Decomposition of the concentration indices and contributions of determinants of under five child stunting and underweight, 2014 and 2022}
\centering
\begin{tabular}[t]{>{\raggedright\arraybackslash}p{2cm}>{\centering\arraybackslash}p{0.8cm}>{\centering\arraybackslash}p{2cm}>{\centering\arraybackslash}p{0.8cm}>{\centering\arraybackslash}p{2cm}>{\centering\arraybackslash}p{0.8cm}>{\centering\arraybackslash}p{2cm}>{\centering\arraybackslash}p{0.8cm}>{\centering\arraybackslash}p{2cm}}
\toprule
\multicolumn{1}{c}{\textbf{ }} & \multicolumn{4}{c}{\textbf{Stunting}} & \multicolumn{4}{c}{\textbf{Underweight}} \\
\cmidrule(l{3pt}r{3pt}){2-5} \cmidrule(l{3pt}r{3pt}){6-9}
\multicolumn{1}{c}{\textbf{ }} & \multicolumn{2}{c}{\textbf{2014}} & \multicolumn{2}{c}{\textbf{2022}} & \multicolumn{2}{c}{\textbf{2014}} & \multicolumn{2}{c}{\textbf{2022}} \\
\cmidrule(l{3pt}r{3pt}){2-3} \cmidrule(l{3pt}r{3pt}){4-5} \cmidrule(l{3pt}r{3pt}){6-7} \cmidrule(l{3pt}r{3pt}){8-9}
  & CI & $\%$ Contribution & CI & $\%$ Contribution & CI & $\%$ Contribution & CI & $\%$ Contribution\\
\midrule
Childs sex & -0.002 & -0.003 & -0.003 & -0.005 & -0.003 & -0.005 & -0.003 & -0.004\\
Residence & -0.540 & -0.332 & -0.617 & -0.634 & -0.617 & -0.636 & -0.617 & -0.636\\
Religion & -0.050 & -0.007 & -0.019 & -0.001 & -0.020 & -0.018 & -0.020 & -0.018\\
Mothers education & 0.390 & 0.226 & 0.449 & -0.165 & 0.449 & 0.589 & 0.449 & 0.589\\
Mothers age (years) & -0.012 & -0.012 & 0.040 & 0.011 & 0.040 & -0.011 & 0.040 & -0.011\\
\addlinespace
Mothers work & 0.078 & -0.021 & 0.180 & -0.021 & 0.180 & 0.005 & 0.180 & 0.005\\
Fathers education & 0.404 & -0.006 & 0.494 & 0.275 & 0.494 & 0.390 & 0.494 & 0.390\\
Delivery place & 0.332 & 0.046 & 0.181 & -0.051 & 0.181 & 0.003 & 0.181 & 0.003\\
Region & 0.131 & 0.034 & 0.122 & 0.088 & 0.122 & 0.084 & 0.122 & -0.026\\
Birth interval (months) & 0.110 & 0.078 & 0.098 & 0.069 & -0.216 & 0.068 & 0.098 & 0.068\\
\addlinespace
Birth order number & -0.154 & 0.163 & -0.144 & 0.057 & -0.144 & -0.029 & -0.144 & -0.029\\
Childs age (months) & 0.003 & -0.003 & -0.002 & 0.006 & -0.002 & 0.003 & -0.002 & 0.003\\
Wealth index & 0.677 & 0.896 & 0.693 & 1.744 & 0.693 & 0.969 & 0.693 & 0.969\\
\bottomrule
\end{tabular}
\end{sidewaystable}

\hypertarget{sec12}{%
\section{Discussion}\label{sec12}}

\hypertarget{sec13}{%
\section{Conclusion}\label{sec13}}

\backmatter

\bmhead{Supplementary information}

\bmhead{Acknowledgments}

\hypertarget{declarations}{%
\section*{Declarations}\label{declarations}}
\addcontentsline{toc}{section}{Declarations}

Some journals require declarations to be submitted in a standardised
format. Please check the Instructions for Authors of the journal to
which you are submitting to see if you need to complete this section. If
yes, your manuscript must contain the following sections under the
heading `Declarations':

\begin{itemize}
\tightlist
\item
  Funding
\item
  Conflict of interest/Competing interests (check journal-specific
  guidelines for which heading to use)
\item
  Ethics approval
\item
  Consent to participate
\item
  Consent for publication
\item
  Availability of data and materials
\item
  Code availability
\item
  Authors' contributions
\end{itemize}

\noindent If any of the sections are not relevant to your manuscript,
please include the heading and write `Not applicable' for that section.

\bibliography{bibliography.bib}


\end{document}
