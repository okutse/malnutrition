%Version 2.1 April 2023
% See section 11 of the User Manual for version history
%
%%%%%%%%%%%%%%%%%%%%%%%%%%%%%%%%%%%%%%%%%%%%%%%%%%%%%%%%%%%%%%%%%%%%%%
%%                                                                 %%
%% Please do not use \input{...} to include other tex files.       %%
%% Submit your LaTeX manuscript as one .tex document.              %%
%%                                                                 %%
%% All additional figures and files should be attached             %%
%% separately and not embedded in the \TeX\ document itself.       %%
%%                                                                 %%
%%%%%%%%%%%%%%%%%%%%%%%%%%%%%%%%%%%%%%%%%%%%%%%%%%%%%%%%%%%%%%%%%%%%%

\documentclass[sn-basic,Numbered,pdflatex]{sn-jnl}

%%%% Standard Packages
%%<additional latex packages if required can be included here>

\usepackage{graphicx}%
\usepackage{multirow}%
\usepackage{amsmath,amssymb,amsfonts}%
\usepackage{amsthm}%
\usepackage{mathrsfs}%
\usepackage[title]{appendix}%
\usepackage{xcolor}%
\usepackage{textcomp}%
\usepackage{manyfoot}%
\usepackage{booktabs}%
\usepackage{algorithm}%
\usepackage{algorithmicx}%
\usepackage{algpseudocode}%
\usepackage{listings}%
%%%%

%%%%%=============================================================================%%%%
%%%%  Remarks: This template is provided to aid authors with the preparation
%%%%  of original research articles intended for submission to journals published
%%%%  by Springer Nature. The guidance has been prepared in partnership with
%%%%  production teams to conform to Springer Nature technical requirements.
%%%%  Editorial and presentation requirements differ among journal portfolios and
%%%%  research disciplines. You may find sections in this template are irrelevant
%%%%  to your work and are empowered to omit any such section if allowed by the
%%%%  journal you intend to submit to. The submission guidelines and policies
%%%%  of the journal take precedence. A detailed User Manual is available in the
%%%%  template package for technical guidance.
%%%%%=============================================================================%%%%

%% Per the spinger doc, new theorem styles can be included using built in style, 
%% but it seems the don't work so commented below
%\theoremstyle{thmstyleone}%
\newtheorem{theorem}{Theorem}%  meant for continuous numbers
%%\newtheorem{theorem}{Theorem}[section]% meant for sectionwise numbers
%% optional argument [theorem] produces theorem numbering sequence instead of independent numbers for Proposition
\newtheorem{proposition}[theorem]{Proposition}%
%%\newtheorem{proposition}{Proposition}% to get separate numbers for theorem and proposition etc.

%% \theoremstyle{thmstyletwo}%
\theoremstyle{remark}
\newtheorem{example}{Example}%
\newtheorem{remark}{Remark}%

%% \theoremstyle{thmstylethree}%
\theoremstyle{definition}
\newtheorem{definition}{Definition}%

\usepackage{booktabs}
\usepackage{longtable}
\usepackage{array}
\usepackage{multirow}
\usepackage{wrapfig}
\usepackage{float}
\usepackage{colortbl}
\usepackage{pdflscape}
\usepackage{tabu}
\usepackage{threeparttable}
\usepackage{threeparttablex}
\usepackage[normalem]{ulem}
\usepackage{makecell}
\usepackage{xcolor}


\raggedbottom




% tightlist command for lists without linebreak
\providecommand{\tightlist}{%
  \setlength{\itemsep}{0pt}\setlength{\parskip}{0pt}}





\begin{document}


\title[Article Title runing]{Socioeconomic disparities in child
malnutrition: An analysis of evidence from Kenya Demographic and Health
Survey, 2014 - 2022}

%%=============================================================%%
%% Prefix	-> \pfx{Dr}
%% GivenName	-> \fnm{Joergen W.}
%% Particle	-> \spfx{van der} -> surname prefix
%% FamilyName	-> \sur{Ploeg}
%% Suffix	-> \sfx{IV}
%% NatureName	-> \tanm{Poet Laureate} -> Title after name
%% Degrees	-> \dgr{MSc, PhD}
%% \author*[1,2]{\pfx{Dr} \fnm{Joergen W.} \spfx{van der} \sur{Ploeg} \sfx{IV} \tanm{Poet Laureate}
%%                 \dgr{MSc, PhD}}\email{iauthor@gmail.com}
%%=============================================================%%

\author*[1,2]{\fnm{Amos
O.} \sur{Okutse} }\email{\href{mailto:amos_okutse@brown.edu}{\nolinkurl{amos\_okutse@brown.edu}}}

\author[2]{\fnm{Henry} \sur{Athiany} }

\author[1]{\fnm{Joseph W.} \sur{Hogan} }



  \affil*[1]{\orgdiv{Department of Biostatistics}, \orgname{Brown
University School of Public
Health}, \orgaddress{\city{Providence}, \country{USA}, \state{RI}, \street{121
South Main St}}}
  \affil[2]{\orgdiv{Department of Math and Statistics}, \orgname{School
of Physical and Mathematical Sciences, Jomo Kenyatta University of
Agriculture and
Technology}, \orgaddress{\city{Nairobi}, \country{Kenya}}}

\abstract{\textbf{Background}: Child malnutrition remains a critical
public health issue, with socioeconomic factors playing a significant
role. Despite overall progress in reducing under-five child malnutrition
in Kenya, disparities persist. This paper explores trends, determinants,
contributions of these factors to health inequality, and their clinical
utility as diagnostic tests for chronic under-five child malnutrition.

\textbf{Methods:} We used data from the Kenya Demographic and Health
Survey (2014 -- 2022) and analyzed malnutrition using three indicators:
stunting, underweight, and wasting. Malnutrition determinants were
analyzed using multivariable logistic regression. Trends in
socioeconomic inequality were analyzed using concentration indices.
Wagstaff decomposition was used to explore the contributions of
determinants of child malnutrition. Diagnostic utility was investigated
using sensitivity, specificity, predictive values, and area under the
ROC curve.

\textbf{Results:} Socioeconomic inequality in under-five child
malnutrition increased between 2014 and 2022 despite an overall decrease
in prevalence. Children from the poorest (AOR = 1.67; 95\% Confidence
Interval {[}CI{]}: 1.24 -- 2.26) and poorer (AOR = 1.46; 95\%CI: 1.08 --
1.98) socioeconomic quintiles are disproportionately affected by
stunting. A child's age (in months) (AOR = 1.01; 95\%CI: 1.01 -- 1.02),
birth order number (AOR = 1.05; 95\%CI: 1.01 -- 1.10), and gender (male)
(AOR = 1.45; 95\%CI: 1.30 -- 1.62) were also significantly associated
with increased odds of stunting. Socioeconomic inequality was the
largest contributing factor to health inequality. Sensitivity,
specificity, and AUC values were 67.4\% (95\% CI: 66.4\% -- 68.4\%),
50.6\% (95\%CI: 50.0\% - 51.1\%), and 0.59 (95\%CI: 0.58 -- 0.60),
respectively when using socioeconomic status as a screening tool for
stunting.

\textbf{Conclusion:} Socioeconomic disparities remain a barrier despite
strides in reducing child malnutrition in Kenya. Targeted interventions
addressing these disparities are essential for sustainable improvements
in child health. Our findings underscore the importance of integrating
socioeconomic factors into public health strategies to combat child
malnutrition effectively.}

\keywords{Child malnutrition, Decomposition, Socioeconomic
inequality, Kenya, Stunting, Underweight, Wasting, Demographic Health
Survey}



\maketitle

\hypertarget{sec1}{%
\section{Introduction}\label{sec1}}

Child Malnutrition remains a dominant public health challenge globally.
In 2022 about 148.1 million (22.3\%) children below 5 years were
stunted, whereas 45 million (6.8\%) and 37 million (5.6\%) were wasted
and overweight, respectively \citep{who2023}. While there has been some
progress in the actualization of the global nutrition targets, this
progress is slow, and the levels of malnutrition continue to persist.
Africa and Asia account for almost half the world's child malnutrition
burden \citep{who2023}.

In East Africa, stunting prevalence (32.6\%) was higher than the global
average (22.3\%), whereas wasting and overweight were 5.2\% and 4.0\%,
respectively \citep{IEG2022}. With the possibility of suffering from
more than one form of malnutrition, children remain largely susceptible
to the perilous effects posed by this condition. According to nutrition
statistics, 3.62\% of all children under the age of five years (15.95
million) have been reported as being both stunted and wasted, whereas
1.87\% of all children (8.23 million) have been reported to experience
both stunting and overweight globally \citep{global}.

In the first half of 2022, Kenya reported about 942,000 cases of acute
malnutrition among children between 6 and 59 months
\citep{Bhavnani2023}. According to the 2022 Kenya Demographic and Health
Survey (KDHS) \citep{KNBSICF2023}, 18\% of children under five are
stunted (chronically undernourished), 5\% are wasted (acutely
malnourished), whereas 3\% and 10\% are overweight and underweight,
respectively. While Kenya has substantially reduced the burden of child
malnutrition, undernutrition is estimated to cost the country over
US\$38.3 billion in Gross Domestic Product (GDP) following losses in
workforce labor and productivity for 2010 -- 2030 \citep{USAID2018}.

Malnutrition denotes ``a state of nutrition in which a deficiency, or an
excess, of energy, protein, and micronutrient causes measurable adverse
effects on tissue/body form (body shape, size, and composition),
function, and clinical outcome'' \citep{stratton2003}. This has been
attributed to several diverse interlinked factors with detrimental short
and long-term effects \citep{Pelletier1995, Pelletier2003}. Not only
does it affect the physical and cognitive development of a child, but it
also drastically increases their risk of infections and contributes
negatively to their mortality and morbidity
\citep{Rice2000, jonah2018, victora2008, kar2008, mendez1999, walker2015, rabbani2016}.

Stunting, underweight, and wasting remain the recommended indicators of
malnutrition \citep{jonah2018}. Stunting refers to low height for age
and reflects the growth in linear terms achieved at the age at which the
anthropometric measurements were taken. Underweight is low weight for
age, resulting from a short-term lack of food. In contrast, wasting is
severe undernutrition resulting from inadequate food intake and
infections \citep{jonah2018}. In children under 5 years, stunting is the
most significant measure of overall health and well-being capable of
highlighting salient social disparities \citep{deOnis2016}. Moreover,
because stunting measures linear growth in children, it is considered an
accurate measure of malnutrition in the long term due to its
insensitivity to variations in food consumption
\citep{Hoddinott2013, Zere2003}.

Kenya is classified as a middle-income country based on its Gross
National Income (GNI) per capita. Under this classification frame,
countries are classified into three categories based on their income as
either low, middle, or high-income countries \citep{jonah2018}. A
country's attainment of the middle-income classification status is often
seen as an indication of progress resulting from such activities as
heightened investments across all government sectors and improved
productivity. Shifts in a country's classification from low to middle,
then to high income, indicate economic advancement. As expected of
growth, such advancements are expected to impact the well-being of a
country's population positively. For instance, economic advancements are
expected to create employment opportunities, translating into increased
disposable income, improved health, and education
\citep{Bloom2004, Ranis2000}. Improved living standards following
economic advancement are expected to translate into improved nutritional
consequences for children and adults \citep{Marmot2002, Ettner1996}.
However, economic advancement does not necessarily translate to
equitable distribution of positive prospects across the population.
Often, these tend to be skewed, with some groups benefiting more than
others.

Kenya has made commendable progress in reducing the burden of
malnutrition as part of the Standard Development Goals (SDGs), which
have considerably reduced the stunting rate. Even so, the overall
prevalence of the condition remains larger than those observed for other
forms of malnutrition \citep{jonah2018, Haddad2015}. Given the danger
malnutrition poses to child growth, survival, and well-being, its
consequences are of substantial interest to the government, public
health professionals, and policymakers.

This study contributes significantly to the available knowledge on
socioeconomic disproportions in the Kenyan child malnutrition burden.
First, we examine trends in stunting, underweight, and wasting across
socioeconomic groups, geography, and selected household, child,
maternal, and paternal characteristics. Second, we explore determinants
of child malnutrition and employ standard procedures of inequality
analysis to quantify their contributions to health inequality in Kenya.
Finally, we examine the independent clinical utility of household
socioeconomic status and significant child characteristics in acute
malnutrition screening. The current study uses data from the Kenya
Demographic Health Survey (DHS) (2014 to 2022) to comprehensively
analyze the scope of the problem, specifically focusing on children
below five years. We provide evidence to inform the design of competent
public health strategies and targeted interventions for curbing
under-five child malnutrition and provide a profound understanding of
socioeconomic inequality and child well-being.

\hypertarget{sec11}{%
\section{Methods}\label{sec11}}

\hypertarget{study-data}{%
\subsection{Study data}\label{study-data}}

We utilized data from the 2014 and 2022 Kenya Demographic and Health
Surveys (KDHS) (standard DHS). These surveys adopt a two-stage
stratified cluster sampling approach with clusters sampled in the first
stage and households at the second stage. In 2014, a response rate of
99\% was achieved from 39,679 households, and in 2022, a 98\% response
rate from 38,731 occupied households \citep{KNBSICF2015, KNBSICF2023}.
Analyses considered all live children (0-59 months) of interviewed
mothers, excluding those with missing anthropometric data. The data was
weighted for non-response and used with DHS authorization.

\hypertarget{variables}{%
\subsection{Variables}\label{variables}}

\hypertarget{outcome}{%
\subsubsection{Outcome}\label{outcome}}

Malnutrition was characterized by stunting (low height-for-age z-scores,
HAZ), underweight (low weight-for-age z-scores, WAZ), and wasting (low
weight-for-height z-scores, WHZ) \citep{kien_trends_2016, jonah2018}.
Stunting indicates a child's linear growth at a given age. Underweight
results from short-term food deprivation, while wasting stems from both
inadequate food intake and infections \citep{jonah2018}.

In children under five, a HAZ, WAZ, or WHZ between -2 and -3 standard
deviations (SD) below the median suggests moderate stunting,
underweight, or wasting, respectively. Z-scores less than -3 SD below
the World Health Organization's (WHO) child growth standards median
indicate severe conditions \citep{WHO2010}.

We categorized children with HAZ, WAZ, and WHZ scores below -2 SD of the
WHO growth standards median as stunted, underweight, or wasted,
respectively. Notably, stunting in children under five highlights
chronic undernutrition, whereas wasting and underweight can imply both
acute and chronic malnutrition \citep{kien_trends_2016}.

\hypertarget{covariates}{%
\subsubsection{Covariates}\label{covariates}}

We considered a comprehensive set of determinants linked to child
malnutrition. Child-specific variables included age (in months), gender,
place and region of residence, delivery location, and birth order. At
the household level, we accounted for religion and socioeconomic status.
Maternal indicators were age, education level, and birth interval
whereas paternal education was considered as a paternal characteristic.
Our choice of these covariates is grounded in existing literature and
availability in our data set
\citep{jonah2018, kien_trends_2016, Farah2019}.

\hypertarget{statistical-analysis}{%
\subsection{Statistical analysis}\label{statistical-analysis}}

\hypertarget{weighted-prevalence-of-child-malnutrition}{%
\subsubsection{Weighted prevalence of child
malnutrition}\label{weighted-prevalence-of-child-malnutrition}}

The weighted prevalence of stunting, underweight, and wasting was
estimated in relation to maternal, child, and household characteristics.
Overall differences across categories were examined using a design-based
Pearson chi-squared test whereas the significance of differences in
group means was analyzed using two-sample t-tests for continuous
variables. The significance of the differences in trends in child
malnutrition by socioeconomic status between 2014 and 2022 was similarly
analyzed using two-sample proportion tests.

\hypertarget{analysis-of-disparities-in-child-malnutrition}{%
\subsubsection{Analysis of disparities in child
malnutrition}\label{analysis-of-disparities-in-child-malnutrition}}

The extent and trends of socioeconomic disparities in stunting,
underweight, and wasting were quantified using concentration indices
(CIs) estimated based on the corresponding z-scores
\citep{odonnell_analyzing_2008, Wagstaff1991, Wagstaff2000}.
Concentration indices quantify socioeconomic disparities in a health
variable and allow assessment of the extent and levels of disparities.
CIs were computed as double the area between the concentration curve and
the line of equality -- the \(45^{\circ}\) line.

According to O'Donnell et al. \citep{odonnell_analyzing_2008}:
\begin{equation}
CI = \frac{2}{\mu} \textrm{cov}(h, r)
\label{eq:one}
\end{equation}

In Equation (\ref{eq:one}), \(\mu\) is the average of malnutrition
(stunting, underweight, and wasting) in children under five children,
\(h\) denotes observation-specific child malnutrition, and \(r\) is the
rank of the socioeconomic status of a household. The CI of a given
health variable usually takes values between -1 and +1, with 0
suggesting perfect equity of the health variable between the poorest and
the richest socioeconomic groups. Negative values suggest a higher
concentration of malnutrition among the poorest group whereas positive
values suggest a higher concentration of inequity among the richest
socioeconomic group
\citep{Akombi2019, jonah2018, kien_trends_2016, Wagstaff2000}.

\hypertarget{analysis-of-determinants-of-child-malnutrition-and-utility-in-screening-for-child-stunting}{%
\subsubsection{Analysis of determinants of child malnutrition and
utility in screening for child
stunting}\label{analysis-of-determinants-of-child-malnutrition-and-utility-in-screening-for-child-stunting}}

Determinants of child malnutrition were investigated using binary
logistic regression. Separate models were fitted for stunting,
underweight, and wasting. Odds ratios were computed for each adjustment
covariate to examine associations between malnutrition indicators and
explanatory variables in \ref{covariates}. Results from the fitted
logistic regression models were used in evaluating the clinical utility
of each significant determinant of under-five child stunting. Our
analyses here focus on this indicator since it suggests chronic
malnutrition \citep{jonah2018}.

We estimated the diagnostic performance of a household's socioeconomic
status and child characteristics, which we found to significantly impact
under-five child stunting, including age, gender, and birth order. These
factors were each used to independently predict a child's nutritional
status (stunting) and compute sensitivity, specificity, predictive
values, and area under the receiver operating characteristic curve
(AUC). We leveraged the \texttt{diagti} command in STATA
\citep{seed2010diagt}. Stunting had a 22.7\% (95\%CI: 22.3\% - 23.1\%)
prevalence in this dataset, suggesting a classification problem with an
imbalanced data set. Evaluating the clinical utility of these risk
factors would result in highly optimistic results characterized by a
high hit ratio but poor explanatory capabilities.

Most classifiers rely on threshold scores to generate predictions, which
greatly impacts the trade-off between positive and negative errors
\citep{provost2008machine}. We used threshold-moving, a technique for
training a cost-sensitive classifier. We adjusted the decision threshold
to accurately predict the minority class and address class imbalance by
setting the optimal threshold based on a random search maximizing the
AUC \citep{haibo2013imbalanced}. The choice of the AUC as a performance
metric was informed by its insensitivity to changes in class
distribution. An AUC of 0.5 suggests limited discriminatory ability of a
test that is no better than random guessing. Values between 0.7 to 0.8
are acceptable, 0.8 to 0.9 are excellent whereas those above 0.9 are
considered exceptional \citep{Fawcett2006, Mandrekar2010}.

\hypertarget{decompose}{%
\subsubsection{Decomposition of socioeconomic inequities in child
malnutrition}\label{decompose}}

Contributions of determinants of malnutrition in children under five to
the observed socioeconomic disparities were examined through a
decomposition analysis. This decomposition was restricted to stunting
and underweight, indicators that exhibited substantial differences
between 2014 and 2022.

We considered a linear regression model where the response variable
(\(y\)) is modeled as a linear combination of the \(k\) determinants
(\(X_k\)) as: \begin{equation}
y = \alpha + \sum_{k} \beta_k X_k + \epsilon
\label{eq:two}
\end{equation} where \(\beta_k\) denotes the coefficient of \(X_k\) and
\(\epsilon\) is the error term.

In terms of the CI for the response \(y\), (\ref{eq:two}) becomes:
\begin{equation}
CI = \sum_{k} \left( \frac{\beta_k \bar{X}_k}{\mu} \right) CI_k + \frac{G CI_{\epsilon}}{\mu}
\label{eq:three}
\end{equation} where \(\mu\) denotes the average of \(y\), \(\bar{X}_k\)
denotes the mean of the \(k^{th}\) variable, \(\beta_k\) denotes the
coefficient of each determinant, \(CI_k\) denotes the CI of each of the
regressors in the model, and \(G CI_\epsilon\) denotes the generalized
concentration index for the error term, \(\epsilon\).

Equation (\ref{eq:three}) has two components: the explained
\(\left( (\beta_k \bar{X}_k)/ \mu)CI_k\right)\) and the unexplained
component (\(GCI_{\epsilon}/ \mu\)). \((\beta_k \bar{X}_k)/\mu\) is the
elasticity denoting the effect of each \$CI\_k \$ on the overall CI of
the outcome variable, \(y\). We employed the Wagstaff normalization
technique for CI values given our use of binary outcomes (the CI bounds
would otherwise not be between -1 and +1)
\citep{Wagstaff2000, Wagstaff2003}.

\hypertarget{sec2}{%
\section{Results}\label{sec2}}

\hypertarget{weighted-prevalence-of-child-malnutrition-1}{%
\subsection{Weighted prevalence of child
malnutrition}\label{weighted-prevalence-of-child-malnutrition-1}}

Table \ref{tab:one} highlights the weighted prevalence of under-five
child malnutrition by selected child, household, maternal, and paternal
characteristics grouped by the survey year (2014 - 2022). The sample
size for the 2014 KDHS was n = 18702 (53\%) and for the 2022 KDHS was n
= 16883 (47\%). In 2014, 26\% (n = 4466) of children were found to be
stunted, 11\% (n = 1841) were underweight, and 4\% (n = 701) were
wasted. In contrast, the percentage of stunted children decreased to
18\% (n = 2665) in 2022, while underweight and wasting decreased to 10\%
(n = 1543) and 5\% (n = 752), respectively. The analyzed sample
consisted of 51\% male and 49\% female children.

In 2014, the prevalence of stunting, underweight, and wasting was
significantly higher among older male children delivered at home in
rural areas, in households with the poorest socioeconomic status, and
identifying as Protestant (\(p < 0.05\)). Most stunted children in this
year were born to mothers between 25 and 34 years of age, with at most a
primary school education (\(p < 0.05\)). We did not find evidence of a
significant association between maternal age, employment status and
child stunting (\(p > 0.05\)). In 2022, stunting prevalence remained
similar, except that it was higher among children delivered in public
hospitals.

The prevalence of child underweight or wasting was also higher among
older male children who were delivered in protestant rural households
with the poorest socio-economic status (\(p < 0.05\)). These children
were mostly born to parents with at most primary education in the Rift
Valley region (\(p < 0.05\)). Similar patterns in the prevalence of
underweight and wasting were observed in 2022, except that most of these
cases were deliveries in hospitals within the public sector.

\renewcommand{\arraystretch}{0.8}

\begin{landscape}\begingroup\fontsize{7}{9}\selectfont

\begin{longtable}[t]{>{\raggedright\arraybackslash}p{1.5cm}>{\centering\arraybackslash}p{0.3cm}>{\centering\arraybackslash}p{1.5cm}>{\centering\arraybackslash}p{0.2cm}>{\centering\arraybackslash}p{0.3cm}>{\centering\arraybackslash}p{1.5cm}>{\centering\arraybackslash}p{0.2cm}>{\centering\arraybackslash}p{0.3cm}>{\centering\arraybackslash}p{1.5cm}>{\centering\arraybackslash}p{0.2cm}>{\centering\arraybackslash}p{0.3cm}>{\centering\arraybackslash}p{1.5cm}>{\centering\arraybackslash}p{0.2cm}>{\centering\arraybackslash}p{0.3cm}>{\centering\arraybackslash}p{1.5cm}>{\centering\arraybackslash}p{0.2cm}>{\centering\arraybackslash}p{0.3cm}>{\centering\arraybackslash}p{1.5cm}>{\centering\arraybackslash}p{0.2cm}}
\caption{\label{tab:one}Weighted prevalence of stunting, underweight, and wasting among children under five years by selected child, household, maternal, and paternal characteristics (2014 -- 2022)}\\
\toprule
\multicolumn{1}{c}{\textbf{ }} & \multicolumn{9}{c}{\textbf{2014}} & \multicolumn{9}{c}{\textbf{2022}} \\
\cmidrule(l{3pt}r{3pt}){2-10} \cmidrule(l{3pt}r{3pt}){11-19}
\multicolumn{1}{c}{\textbf{ }} & \multicolumn{3}{c}{\textbf{Stunted}} & \multicolumn{3}{c}{\textbf{Underweight}} & \multicolumn{3}{c}{\textbf{Wasted}} & \multicolumn{3}{c}{\textbf{Stunted)}} & \multicolumn{3}{c}{\textbf{Underweight}} & \multicolumn{3}{c}{\textbf{Wasted}} \\
\multicolumn{1}{c}{ } & \multicolumn{3}{c}{(HAZ$<$-2SD)} & \multicolumn{3}{c}{(WAZ$<$-2SD)} & \multicolumn{3}{c}{(WHZ$<$-2SD)} & \multicolumn{3}{c}{(HAZ$<$-2SD)} & \multicolumn{3}{c}{(WAZ$<$-2SD)} & \multicolumn{3}{c}{(WHZ$<$-2SD)} \\
\cmidrule(l{0pt}r{0pt}){2-4} \cmidrule(l{0pt}r{0pt}){5-7} \cmidrule(l{0pt}r{0pt}){8-10} \cmidrule(l{0pt}r{0pt}){11-13} \cmidrule(l{0pt}r{0pt}){14-16} \cmidrule(l{0pt}r{0pt}){17-19}
\textbf{ } & \textbf{$n$} & \textbf{$n(\%)$} & \textbf{$P$} & \textbf{$n$} & \textbf{$n(\%)$} & \textbf{$P$} & \textbf{$n$} & \textbf{$n(\%)$} & \textbf{$P$} & \textbf{$n$} & \textbf{$n(\%)$} & \textbf{$P$} & \textbf{$n$} & \textbf{$n(\%)$} & \textbf{$P$} & \textbf{$n$} & \textbf{$n(\%)$} & \textbf{$P$}\\
\midrule
\endfirsthead
\caption[]{(continued)}\\
\toprule
\textbf{ } & \textbf{$n$} & \textbf{$n(\%)$} & \textbf{$P$} & \textbf{$n$} & \textbf{$n(\%)$} & \textbf{$P$} & \textbf{$n$} & \textbf{$n(\%)$} & \textbf{$P$} & \textbf{$n$} & \textbf{$n(\%)$} & \textbf{$P$} & \textbf{$n$} & \textbf{$n(\%)$} & \textbf{$P$} & \textbf{$n$} & \textbf{$n(\%)$} & \textbf{$P$}\\
\midrule
\endhead

\endfoot
\bottomrule
\multicolumn{19}{l}{\rule{0pt}{1em}\textit{Note: } HAZ: height-for-age z-score; WAZ: weight-for-age z-score; WHZ: weight-for-height z-score; SD: standard deviation;}\\
\multicolumn{19}{l}{\rule{0pt}{1em}\textsuperscript{*} $P$ based on a Pearson chi-square test for categorical variable and T-test for continuous variables.}\\
\endlastfoot
Weighted n & 17291 & 4466 (25.8) &  & 17291 & 1841 (10.6) &  & 17291 & 701 (4.1) &  & 15336 & 2665 (17.5) &  & 15368 & 1543 (10.0) &  & 15329 & 752 (4.9) & \\
Child age, mean (SE) &  & 30.8 (0.2) & 0.00 &  & 31.8 (0.5) & 0.00 &  & 26.1 (1.0) & 0.00 &  & 27.7 (0.4) & 0.08 &  & 30.8 (0.5) & 0.00 &  & 30.8 (0.7) & 0.00\\
Birth interval, mean (SE) &  & 39.5 (0.5) & 0.00 &  & 38.2 (0.7) & 0.00 &  & 39.8 (1.2) & 0.00 &  & 43.8 (0.8) & 0.00 &  & 41.8 (1.0) & 0.00 &  & 43.1 (1.5) & 0.00\\
Birth order , mean (SE) &  & 3.6 (0.1) & 0.00 &  & 3.8 (0.1) & 0.00 &  & 3.6 (0.1) & 0.01 &  & 3 (0.1) & 0.00 &  & 3.5 (0.1) & 0.00 &  & 3.5 (0.09) & \\
\textbf{Child sex} & \textbf{} & \textbf{} & \textbf{0.00} & \textbf{} & \textbf{} & \textbf{0.00} & \textbf{} & \textbf{} & \textbf{0.09} & \textbf{} & \textbf{} & \textbf{0.00} & \textbf{} & \textbf{} & \textbf{} & \textbf{} & \textbf{} & \textbf{0.01}\\
\addlinespace
\hspace{1em}Male & 8763 & 2586 (57.9) &  & 8763 & 1028 (55.9) &  & 8763 & 382 (54.6) &  & 7755 & 1523 (57.2) &  & 7767 & 857 (55.6) &  & 7758 & 420 (55.9) & \\
\hspace{1em}Female & 8528 & 1880 (42.1) &  & 8528 & 812 (44.1) &  & 8528 & 318 (45.4) &  & 7581 & 1142 (42.8) &  & 7601 & 685 (44.4) &  & 7571 & 331 (44.1) & \\
\textbf{Delivery place} & \textbf{} & \textbf{} & \textbf{0.00} & \textbf{} & \textbf{} & \textbf{0.00} & \textbf{} & \textbf{} & \textbf{0.00} & \textbf{} & \textbf{} & \textbf{0.00} & \textbf{} & \textbf{} & \textbf{0.00} & \textbf{} & \textbf{} & \textbf{0.00}\\
\hspace{1em}Home & 6513 & 2157 (48.5) &  & 6512 & 1033 (56.4) &  & 6512 & 390 (55.9) &  & 1110 & 302 (16.9) &  & 1116 & 205 (23.4) &  & 1116 & 104 (25.7) & \\
\hspace{1em}Public & 7919 & 1749 (39.3) &  & 7919 & 629 (34.3) &  & 7919 & 235 (33.6) &  & 6095 & 1147 (64.2) &  & 6114 & 513 (58.6) &  & 6094 & 217 (53.1) & \\
\addlinespace
\hspace{1em}Private & 2637 & 495 (11.1) &  & 2637 & 153 (8.4) &  & 2637 & 68 (9.7) &  & 2217 & 321 (18.0) &  & 2217 & 150 (17.2) &  & 2202 & 80 (53.1) & \\
\hspace{1em}Other & 176 & 50 (1.1) &  & 176 & 16 (0.9) &  & 176 & 5 (0.7) &  & 50 & 14 (0.8) &  & 50 & 7 (0.8) &  & 49 & 5 (1.4) & \\
\textbf{Residence} & \textbf{} & \textbf{} & \textbf{0.00} & \textbf{} & \textbf{} & \textbf{0.00} & \textbf{} & \textbf{} & \textbf{0.03} & \textbf{} & \textbf{} & \textbf{0.00} & \textbf{} & \textbf{} & \textbf{0.00} & \textbf{} & \textbf{} & \textbf{0.01}\\
\hspace{1em}Urban & 5927 & 1168 (26.1) &  & 5926 & 397 (21.6) &  & 5926 & 200 (28.6) &  & 5411 & 662 (24.8) &  & 5424 & 361 (23.4) &  & 5410 & 215 (28.7) & \\
\hspace{1em}Rural & 11364 & 3298 (73.9) &  & 11364 & 1443 (78.4) &  & 11364 & 500 (71.4) &  & 9924 & 2003 (75.2) &  & 9944 & 1182 (76.6) &  & 9918 & 536 (71.3) & \\
\addlinespace
\textbf{Religion} & \textbf{} & \textbf{} & \textbf{0.00} & \textbf{} & \textbf{} & \textbf{0.00} & \textbf{} & \textbf{} & \textbf{0.00} & \textbf{} & \textbf{} & \textbf{0.01} & \textbf{} & \textbf{} & \textbf{0.03} & \textbf{} & \textbf{} & \textbf{0.00}\\
\hspace{1em}Catholic & 3097 & 728 (16.3) &  & 3097 & 296 (16.1) &  & 3097 & 126 (18.1) &  & 2670 & 445 (16.7) &  & 2676 & 276 (17.9) &  & 2667 & 128 (17.1) & \\
\hspace{1em}Protestant & 12228 & 3200 (71.8) &  & 12228 & 1257 (68.4) &  & 12228 & 444 (63.5) &  & 10527 & 1864 (70.0) &  & 10541 & 1012 (65.6) &  & 10513 & 432 (57.5) & \\
\hspace{1em}Muslim & 1440 & 346 (7.8) &  & 1440 & 195 (10.6) &  & 1440 & 106 (15.2) &  & 1485 & 232 (8.7) &  & 1489 & 190 (12.3) &  & 1494 & 159 (21.2) & \\
\hspace{1em}Atheist & 457 & 181 (4.1) &  & 457 & 79 (4.3) &  & 457 & 18 (2.6) &  & 213 & 59 (2.2) &  & 217 & 24 (1.6) &  & 213 & 12 (1.7) & \\
\addlinespace
\hspace{1em}Other & 42 & 5.4 (0.1) &  & 42 & 8.5 (0.5) &  & 42 & 4 (0.7) &  & 439 & 63 (2.4) &  & 442 & 40 (2.6) &  & 440 & 19 (2.5) & \\
\textbf{Economic status} & \textbf{} & \textbf{} & \textbf{0.00} & \textbf{} & \textbf{} & \textbf{0.00} & \textbf{} & \textbf{} & \textbf{0.00} & \textbf{} & \textbf{} & \textbf{0.00} & \textbf{} & \textbf{} & \textbf{0.00} & \textbf{} & \textbf{} & \textbf{0.00}\\
\hspace{1em}Poorest & 4178 & 1489 (33.4) &  & 4178 & 792 (43.0) &  & 4178 & 303 (43.2) &  & 3583 & 986 (37.0) &  & 3596 & 679 (44.0) &  & 3583 & 340 (45.2) & \\
\hspace{1em}Poorer & 3631 & 1099 (24.6) &  & 3631 & 435 (23.7) &  & 3631 & 116 (16.6) &  & 2840 & 598 (22.4) &  & 2846 & 286 (18.5) &  & 2849 & 88 (11.8) & \\
\hspace{1em}Middle & 3182 & 808 (18.1) &  & 3182 & 286 (15.6) &  & 3182 & 117 (16.8) &  & 2703 & 439 (16.5) &  & 2707 & 249 (16.1) &  & 2701 & 116 (15.5) & \\
\addlinespace
\hspace{1em}Richer & 2969 & 620 (13.9) &  & 2969 & 204 (11.1) &  & 2969 & 76 (11.0) &  & 3052 & 354 (13.3) &  & 3062 & 190 (12.4) &  & 3040 & 124 (16.5) & \\
\hspace{1em}Richest & 3330 & 446 (10.0) &  & 3330 & 122 (6.7) &  & 3330 & 86 (12.4) &  & 3155 & 286 (10.7) &  & 3154 & 137 (8.9) &  & 3154 & 82 (10.9) & \\
\textbf{Mothers education} & \textbf{} & \textbf{} & \textbf{0.00} & \textbf{} & \textbf{} & \textbf{0.00} & \textbf{} & \textbf{} & \textbf{0.00} & \textbf{} & \textbf{} & \textbf{0.00} & \textbf{} & \textbf{} & \textbf{0.00} & \textbf{} & \textbf{} & \textbf{0.00}\\
\hspace{1em}None & 2057 & 628 (14.1) &  & 2057 & 421 (22.9) &  & 2057 & 210 (30.0) &  & 1606 & 354 (13.3) &  & 1614 & 355 (23.0) &  & 1611 & 248 (33.1) & \\
\hspace{1em}Primary & 9735 & 2890 (64.7) &  & 9735 & 1111.7 (60.4) &  & 9735 & 329 (47.0) &  & 5820 & 1283 (48.2) &  & 5834 & 688 (44.6) &  & 5829 & 257 (34.2) & \\
\addlinespace
\hspace{1em}Higher & 5497 & 946 (21.2) &  & 5497 & 307 (16.7) &  & 5497 & 161.7 (23.1) &  & 7909 & 1027 (38.6) &  & 7919 & 500 (32.4) &  & 7888 & 246 (32.7) & \\
\textbf{Mothers age} & \textbf{} & \textbf{} & \textbf{0.20} & \textbf{} & \textbf{} & \textbf{0.01} & \textbf{} & \textbf{} & \textbf{0.35} & \textbf{} & \textbf{} & \textbf{0.00} & \textbf{} & \textbf{} & \textbf{0.54} & \textbf{} & \textbf{} & \textbf{0.11}\\
\hspace{1em}under 24 & 5000 & 1349 (30.2) &  & 5000 & 474 (25.8) &  & 5000 & 182 (26.1) &  & 4084 & 817 (30.7) &  & 4094 & 390 (25.3) &  & 4083 & 173 (23.1) & \\
\hspace{1em}25 - 34 & 8855 & 2228 (49.9) &  & 8855 & 946 (51.4) &  & 8855 & 375 (53.6) &  & 7852 & 1313 (49.3) &  & 7874 & 800 (51.8) &  & 7855 & 394 (52.4) & \\
\hspace{1em}35+ & 3435 & 888 (19.9) &  & 3435 & 420 (22.8) &  & 3435 & 142 (20.3) &  & 3400 & 534 (20.1) &  & 3400 & 353 (22.9) &  & 3390 & 184 (24.5) & \\
\addlinespace
\textbf{Mother employed} & \textbf{} & \textbf{} & \textbf{0.54} & \textbf{} & \textbf{} & \textbf{0.32} & \textbf{} & \textbf{} & \textbf{0.00} & \textbf{} & \textbf{} & \textbf{0.20} & \textbf{} & \textbf{} & \textbf{0.00} & \textbf{} & \textbf{} & \textbf{0.00}\\
\hspace{1em}No & 3041 & 770 (35.9) &  & 3041 & 342 (38.6) &  & 3041 & 149 (49.0) &  & 7596 & 1362 (51.1) &  & 7625 & 853 (55.3) &  & 7602 & 434 (57.8) & \\
\hspace{1em}Yes & 5263 & 1378 (64.1) &  & 5263 & 545 (61.4) &  & 5263 & 155 (51.0) &  & 7739 & 1302 (48.9) &  & 7742 & 689 (44.7) &  & 7726 & 317 (42.2) & \\
\textbf{Fathers education} & \textbf{} & \textbf{} & \textbf{0.00} & \textbf{} & \textbf{} & \textbf{0.00} & \textbf{} & \textbf{} & \textbf{0.00} & \textbf{} & \textbf{} & \textbf{0.00} & \textbf{} & \textbf{} & \textbf{0.00} & \textbf{} & \textbf{} & \textbf{0.00}\\
\hspace{1em}None & 727 & 213 (10.7) &  & 727 & 155 (18.5) &  & 727 & 81 (27.6) &  & 1257 & 304 (14.5) &  & 1263 & 292 (23.8) &  & 1260 & 188 (30.4) & \\
\addlinespace
\hspace{1em}Primary & 3964 & 1183 (59.3) &  & 3964 & 453 (53.9) &  & 3964 & 120 (40.8) &  & 4637 & 961 (45.8) &  & 4651 & 517 (42.0) &  & 4647 & 217 (35.1) & \\
\hspace{1em}Higher & 3019 & 599 (30.0) &  & 3019 & 232 (27.6) &  & 3019 & 93 (31.6) &  & 6629 & 833 (39.7) &  & 6638 & 419 (34.1) &  & 6615 & 213 (34.5) & \\
\textbf{Region} & \textbf{} & \textbf{} & \textbf{0.00} & \textbf{} & \textbf{} & \textbf{0.00} & \textbf{} & \textbf{} & \textbf{} & \textbf{} & \textbf{} & \textbf{0.00} & \textbf{} & \textbf{} & \textbf{0.00} & \textbf{} & \textbf{} & \textbf{0.00}\\
\hspace{1em}Coast & 1774 & 532 (11.9) &  & 1774 & 226 (12.3) &  & 1774 & 75 (10.8) &  & 1441 & 204 (13.2) &  & 1441 & 204 (13.2) &  & 1430 & 91 (12.2) & \\
\hspace{1em}N.Eastern & 557 & 134 (3.0) &  & 557 & 100 (5.5) &  & 557 & 72 (10.3) &  & 562 & 104 (6.8) &  & 562 & 104 (6.8) &  & 562 & 99 (13.2) & \\
\addlinespace
\hspace{1em}Eastern & 2147 & 640 (14.3) &  & 2147 & 259 (14.1) &  & 2147 & 97 (13.9) &  & 1857 & 206 (13.4) &  & 1857 & 206 (13.4) &  & 1850 & 108 (14.4) & \\
\hspace{1em}Central & 1605 & 289 (6.5) &  & 1605 & 78 (4.3) &  & 1605 & 33 (4.7) &  & 1737 & 94 (6.1) &  & 1737 & 94 (6.1) &  & 1730 & 46 (6.2) & \\
\hspace{1em}R. Valley & 5047 & 1502 (33.7) &  & 5047 & 772 (41.9) &  & 5047 & 289 (41.3) &  & 4768 & 623 (40.4) &  & 4768 & 623 (40.4) &  & 4763 & 287 (38.2) & \\
\hspace{1em}Western & 2031 & 506 (11.3) &  & 2031 & 164 (8.9) &  & 2031 & 41 (6.0) &  & 1514 & 117 (7.6) &  & 1514 & 117 (7.6) &  & 1508 & 31 (4.1) & \\
\hspace{1em}Nyanza & 2448 & 556 (12.5) &  & 2448 & 183 (10.0) &  & 2448 & 47 (6.8) &  & 1877 & 106 (6.9) &  & 1877 & 106 (6.9) &  & 1874 & 44 (5.9) & \\
\addlinespace
\hspace{1em}Nairobi & 1678 & 302 (6.8) &  & 1678 & 56 (3.0) &  & 1678 & 43 (6.1) &  & 1610 & 86 (5.6) &  & 1610 & 86 (5.6) &  & 1609 & 43 (5.8) & \\*
\end{longtable}
\endgroup{}
\end{landscape}
\renewcommand{\arraystretch}{1}

\hypertarget{trends-in-child-malnutrition-and-socioeconomic-inequality}{%
\subsection{Trends in child malnutrition and socioeconomic
inequality}\label{trends-in-child-malnutrition-and-socioeconomic-inequality}}

Table \ref{tab:two} summarizes the prevalence of child malnutrition by
the household socioeconomic status between 2014 and 2022. Stunting and
underweight decreased substantially during this period. The absolute
reduction was 9.1 and 1.7\% for stunting and wasting, respectively.
Underweight prevalence showed the smallest decline between 2014 and 2022
(0.6\%). A more detailed examination revealed a substantial decline in
underweight only in the poorest socioeconomic status quintile compared
to stunting, where significant reductions occurred across all the
socioeconomic status groups.

\begin{table}[!h]

\caption{\label{tab:two}Malnutrition prevalence by household socioeconomic status, \% (SE)}
\centering
\begin{tabular}[t]{lllllll}
\toprule
\textbf{ } & \textbf{Poorest} & \textbf{Poorer} & \textbf{Middle} & \textbf{Richer} & \textbf{Richest} & \textbf{All}\\
\midrule
\addlinespace[0.3em]
\multicolumn{7}{l}{\textbf{Stunting (height for age $<$ -2 SD)}}\\
\hspace{1em}2014 & 34.2 (0.6) & 30.2 (0.7) & 24.9 (0.8) & 20.6 (0.7) & 12.9 (0.7) & 27.1 (0.3)\\
\hspace{1em}2022 & 25.6 (0.6) & 20.5 (0.7) & 15.4 (0.7) & 11.7 (0.6) & 07.7 (0.6) & 18.0 (0.3)\\
\hspace{1em}Diff-1 & 08.6 (0.8)* & 09.8 (1.0)* & 09.4 (1.0)* & 08.9 (1.0)* & 05.2 (0.9)* & 09.1 (0.4)*\\
\addlinespace[0.3em]
\multicolumn{7}{l}{\textbf{Underweight (weight for age $<$ -2 SD)}}\\
\hspace{1em}2014 & 21.2 (0.5) & 12.7 (0.5) & 09.3 (0.5) & 07.4 (0.5) & 04.1 (0.4) & 13.2 (0.2)\\
\hspace{1em}2022 & 21.8 (0.5) & 10.6 (0.5) & 09.6 (0.5) & 06.2 (0.4) & 04.5 (0.4) & 12.6 (0.3)\\
\hspace{1em}Diff-2 & -00.6 (0.7) & 02.0 (0.8)* & -00.3 (0.7) & 01.2 (0.7) & -00.3 (0.6) & 00.6 (0.3)\\
\addlinespace[0.3em]
\multicolumn{7}{l}{\textbf{Wasting (weight for height $<$ -2 SD)}}\\
\hspace{1em}2014 & 09.4 (0.4) & 03.6 (0.3) & 03.8 (0.3) & 03.2 (0.3) & 02.9 (0.3) & 05.5 (0.2)\\
\hspace{1em}2022 & 12.9 (0.4) & 04.2 (0.4) & 05.3 (0.4) & 04.3 (0.4) & 02.9 (0.3) & 07.2 (0.2)\\
\hspace{1em}Diff-3 & -03.5 (0.6)* & -00.6 (0.4) & -01.6 (5.3)* & -01.1 (0.5) & 00.0 (0.5) & -01.7 (0.3)*\\
\bottomrule
\multicolumn{7}{l}{\rule{0pt}{1em}\textit{Note: }}\\
\multicolumn{7}{l}{\rule{0pt}{1em}Diff-1, Diff-2, Diff-3: difference in under five stunting, underweight, and wasting, respectively.}\\
\multicolumn{7}{l}{\rule{0pt}{1em}SE: standard error; SD: standard deviation}\\
\multicolumn{7}{l}{\rule{0pt}{1em}\textsuperscript{*} significance based on two-sample comparisons of differences in proportions}\\
\end{tabular}
\end{table}

Table \ref{tab:three} presents the concentration indices of under five
child malnutrition. The CIs of stunting and underweight were
significantly different from 0 between 2014 and 2022 (\(p < 0.05\)),
whereas wasting did not show a significant difference during this period
(\(p > 0.05\)). All differences in CIs were negative, suggesting that
children from the poorest socioeconomic groups are more likely to be
stunted, underweight, or wasted relative to those from the richest
households. Additionally, absolute values of the CIs of stunting and
underweight in 2022 were higher than in 2014, suggesting that
inequalities in under-five child underweight and stunting increased
during this period. On the other hand, the CI for wasting in 2022 was
lower relative to 2014, suggesting that the inequality in child wasting
declined during this period. The difference in the CI for wasting
between 2014 and 2022 was, however, not significant (\(p > 0.05\)).

\begin{table}[!h]

\caption{\label{tab:three}Under five child malnutrition concentration indices (CI), 2014 -- 2022}
\centering
\begin{tabular}[t]{lcccccc}
\toprule
\multicolumn{1}{c}{\textbf{ }} & \multicolumn{2}{c}{\textbf{Stunted}} & \multicolumn{2}{c}{\textbf{Underweight}} & \multicolumn{2}{c}{\textbf{Wasted}} \\
\cmidrule(l{3pt}r{3pt}){2-3} \cmidrule(l{3pt}r{3pt}){4-5} \cmidrule(l{3pt}r{3pt}){6-7}
\multicolumn{1}{c}{\textbf{ }} & \multicolumn{2}{c}{\textbf{(HAZ $<$ -2 SD)}} & \multicolumn{2}{c}{\textbf{(WAZ $<$ -2 SD)}} & \multicolumn{2}{c}{\textbf{(WHZ $<$ -2 SD)}} \\
\cmidrule(l{3pt}r{3pt}){2-3} \cmidrule(l{3pt}r{3pt}){4-5} \cmidrule(l{3pt}r{3pt}){6-7}
  & CI (SE) & $p*$ & CI (SE) & $p*$ & CI (SE) & $p*$\\
\midrule
Year 2014 & -0.15 (0.01) & 0.00 & -0.27 (0.02) & 0.00 & 12.37 (22.61) & 0.58\\
Year 2022 & -0.79 (0.01) & 0.00 & -0.88 (0.01) & 0.00 & -1.96 (0.05) & 0.00\\
Diff & -0.64 (0.01) & 0.00 & -0.61 (0.02) & 0.00 & -14.33 (22.62) & 0.53\\
\bottomrule
\multicolumn{7}{l}{\rule{0pt}{1em}\textit{Note: }}\\
\multicolumn{7}{l}{\rule{0pt}{1em}Diff: difference in child malnutrition concentration indices between 2014 and 2022;}\\
\multicolumn{7}{l}{\rule{0pt}{1em}SE: standard error; SD: standard deviation; HAZ: height-for-age Z-score;}\\
\multicolumn{7}{l}{\rule{0pt}{1em}WAZ: weight-for-age Z-score; WHZ: weight-for-height Z-score}\\
\multicolumn{7}{l}{\rule{0pt}{1em}\textsuperscript{*} $p$-value based on a two-tailed independence test.}\\
\end{tabular}
\end{table}

\hypertarget{determinants-of-child-malnutrition}{%
\subsection{Determinants of child
malnutrition}\label{determinants-of-child-malnutrition}}

Table \ref{tab:four} presents a summary of the determinants of under
five child stunting, underweight, and wasting based on the multiple
logistic regression. Results are based on analysis of the aggregate 2014
and 2022 KDHS datasets. We found that a child's age (in months) (AOR =
1.01; 95\%CI: 1.01 -- 1.02), birth order number (AOR = 1.05; 95\%CI:
1.01 -- 1.10), gender (male vs female) (AOR = 1.45; 95\%CI: 1.30 --
1.62), and household's socioeconomic status were significantly
associated with increased odds of stunting. The odds of stunting were
substantially higher for children from households in the poorest (AOR =
1.67; 95\%CI: 1.24 -- 2.26) and poorer (AOR = 1.46; 95\%CI: 1.08 --
1.98) socioeconomic quintiles relative to children from households in
the wealthiest socioeconomic status group.

Additionally, we found that the risk of underweight and wasting varied
depending on the child's sex and mother's age. Male children were more
likely to be underweight (AOR = 1.35; 95\%CI: 1.17 -- 1.55) and wasted
(AOR = 1.22; 95\%CI: 1.00 -- 1.48) than female children. Similarly,
children born to mothers aged above 35 years were more likely to be
underweight (AOR = 1.49; 95\%CI: 1.08 -- 2.05) and wasted (AOR = 1.60;
95\%CI: 1.04 -- 2.46). On the other hand, older children were more
likely to be underweight than younger children (AOR = 1.01; 95\%CI: 1.00
-- 1.01).

\renewcommand{\arraystretch}{0.8}
\begin{table}

\caption{\label{tab:four}Determinants of under five child malnutrition, KDHS 2014 -- 2022}
\centering
\begin{tabular}[t]{>{\raggedright\arraybackslash}p{2cm}llllll}
\toprule
\multicolumn{1}{c}{\textbf{ }} & \multicolumn{2}{c}{\textbf{Stunted}} & \multicolumn{2}{c}{\textbf{Underweight}} & \multicolumn{2}{c}{\textbf{Wasted}} \\
\multicolumn{1}{c}{ } & \multicolumn{2}{c}{(HAZ$<$-2SD)} & \multicolumn{2}{c}{(WAZ$<$-2SD)} & \multicolumn{2}{c}{(WHZ$<$-2SD)} \\
\cmidrule(l{0pt}r{0pt}){2-3} \cmidrule(l{0pt}r{0pt}){4-5} \cmidrule(l{0pt}r{0pt}){6-7}
\textbf{ } & \textbf{AOR 95$\%$ CI} & \textbf{$p$} & \textbf{AOR 95$\%$ CI} & \textbf{$p$} & \textbf{AOR 95$\%$ CI} & \textbf{$p$}\\
\midrule
Year 2014 & 1.04 (0.90 - 1.19) & 0.60 & 0.74 (0.63 - 0.88) & 0.00 & 0.71 (0.56 - 0.91) & 0.01\\
Child age (months) & 1.01 (1.01 - 1.02) & 0.00 & 1.01 (1.00 - 1.01) & 0.00 & 0.99 (0.98 - 0.99) & 0.00\\
Birth interval & 0.99 (0.99 - 1.00) & 0.00 & 0.99 (0.99 - 1.00) & 0.00 & 0.93 (0.87 - 1.00) & 0.04\\
Birth order number & 1.05 (1.01 - 1.10) & 0.01 & 0.96 (0.91 - 1.00) & 0.07 & 0.99 (0.98 - 0.99) & 0.00\\
\textbf{Childs sex} & \textbf{} & \textbf{} & \textbf{} & \textbf{} & \textbf{} & \textbf{}\\
\addlinespace
\hspace{1em}Male & 1.45 (1.30 - 1.62) & 0.00 & 1.35 (1.17 - 1.55) & 0.00 & 1.22 (1.00 - 1.48) & 0.05\\
\hspace{1em}Female & ref &  & ref &  & ref & \\
\textbf{Delivery place} & \textbf{} & \textbf{} & \textbf{} & \textbf{} & \textbf{} & \textbf{}\\
\hspace{1em}Home & ref &  & ref &  & ref & \\
\hspace{1em}Public & 0.81 (0.70 - 0.94) & 0.01 & 0.67 (0.56 - 0.81) & 0.00 & 0.71 (0.54 - 0.94) & 0.02\\
\addlinespace
\hspace{1em}Private & 0.85 (0.69 - 1.06) & 0.15 & 0.67 (0.50 - 0.89) & 0.01 & 0.75 (0.50 - 1.14) & 0.18\\
\hspace{1em}Other & 1.18 (0.72 - 1.93) & 0.50 & 1.19 (0.63 - 2.25) & 0.59 & 1.96 (0.79 - 4.87) & 0.15\\
\textbf{Residence} & \textbf{} & \textbf{} & \textbf{} & \textbf{} & \textbf{} & \textbf{}\\
\hspace{1em}Urban & 0.88 (0.74 - 1.03) & 0.11 & 0.79 (0.64 - 0.99) & 0.04 & 1.01 (0.76 - 1.34) & 0.96\\
\hspace{1em}Rural & ref &  & ref &  & ref & \\
\addlinespace
\textbf{Religion} & \textbf{} & \textbf{} & \textbf{} & \textbf{} & \textbf{} & \textbf{}\\
\hspace{1em}Catholic & 0.42 (0.31 - 0.58) & 0.00 & 0.54 (0.38 - 0.78) & 0.00 & 0.39 (0.26 - 0.59) & 0.00\\
\hspace{1em}Protestant & 0.47 (0.35 - 0.63) & 0.00 & 0.53 (0.38 - 0.73) & 0.00 & 0.48 (0.33 - 0.71) & 0.00\\
\hspace{1em}Muslim & 0.35 (0.24 - 0.51) & 0.00 & 0.34 (0.22 - 0.53) & 0.00 & 0.41 (0.24 - 0.69) & 0.00\\
\hspace{1em}Other & 0.47 (0.26 - 0.83) & 0.01 & 0.59 (0.31 - 1.13) & 0.11 & 0.54 (0.24 - 1.20) & 0.13\\
\addlinespace
\hspace{1em}Atheist & ref &  & ref &  & ref & \\
\textbf{Economic status} & \textbf{} & \textbf{} & \textbf{} & \textbf{} & \textbf{} & \textbf{}\\
\hspace{1em}Poorest & 1.67 (1.24 - 2.26) & 0.00 & 1.08 (0.74 - 1.56) & 0.69 & 0.90 (0.58 - 1.42) & 0.66\\
\hspace{1em}Poorer & 1.46 (1.08 - 1.98) & 0.01 & 0.80 (0.56 - 1.16) & 0.24 & 0.51 (0.31 - 0.82) & 0.01\\
\hspace{1em}Middle & 1.32 (0.98 - 1.79) & 0.07 & 0.85 (0.59 - 1.23) & 0.39 & 1.05 (0.65 - 1.69) & 0.85\\
\addlinespace
\hspace{1em}Richer & 1.23 (0.91 - 1.66) & 0.07 & 0.76 (0.53 - 1.11) & 0.16 & 0.85 (0.55 - 1.31) & 0.46\\
\hspace{1em}Richest & ref &  &  &  & ref & \\
\textbf{Mothers education} & \textbf{} & \textbf{} & \textbf{} & \textbf{} & \textbf{} & \textbf{}\\
\hspace{1em}None & ref &  & ref &  & ref \vphantom{1} & \\
\hspace{1em}Primary & 1.12 (0.92 - 1.37) & 0.27 & 0.82 (0.65 - 1.03) & 0.08 & 0.61 (0.45 - 0.83) & 0.00\\
\addlinespace
\hspace{1em}Higher & 0.81 (0.64 - 1.04) & 0.10 & 0.46 (0.34 - 0.63) & 0.00 & 0.42 (0.27 - 0.65) & 0.00\\
\textbf{Mother's age (years)} & \textbf{} & \textbf{} & \textbf{} & \textbf{} & \textbf{} & \textbf{}\\
\hspace{1em}Under 24 & ref &  & ref &  & ref & \\
\hspace{1em}25 - 34 & 0.78 (0.66 - 0.93) & 0.00 & 1.34 (1.08 - 1.66) & 0.01 & 1.25 (0.94 - 1.66) & 0.13\\
\hspace{1em}35+ & 0.65 (0.50 - 0.84) & 0.00 & 1.49 (1.08 - 2.05) & 0.00 & 1.60 (1.04 - 2.46) & 0.03\\
\addlinespace
\textbf{Mother employed} & \textbf{} & \textbf{} & \textbf{} & \textbf{} & \textbf{} & \textbf{}\\
\hspace{1em}No & ref &  & ref &  & ref & \\
\hspace{1em}Yes & 1.06 (0.93 - 1.21) & 0.37 & 1.02 (0.87 - 1.21) & 0.80 & 0.91 (0.72 - 1.15) & 0.44\\
\textbf{Fathers education} & \textbf{} & \textbf{} & \textbf{} & \textbf{} & \textbf{} & \textbf{}\\
\hspace{1em}None & ref &  & ref &  & ref & \\
\addlinespace
\hspace{1em}Primary & 0.91 (0.75 - 1.11) & 0.36 & 0.67 (0.53 - 0.85) & 0.00 & 0.64 (0.46 - 0.89) & 0.01\\
\hspace{1em}Higher & 0.75 (0.59 - 0.94) & 0.01 & 0.65 (0.49 - 0.86) & 0.00 & 0.64 (0.43 - 0.96) & 0.03\\
\textbf{Region} & \textbf{} & \textbf{} & \textbf{} & \textbf{} & \textbf{} & \textbf{}\\
\hspace{1em}Coast & 0.62 (0.41 - 0.93) & 0.02 & 0.85 (0.54 - 1.33) & 0.48 & 0.70 (0.39 - 1.26) & 0.23\\
\hspace{1em}N.eastern & 0.34 (0.22 - 0.55) & 0.00 & 0.82 (0.52 - 1.29) & 0.38 & 1.29 (0.71 - 2.36) & 0.40\\
\addlinespace
\hspace{1em}Eastern & 0.62 (0.42 - 0.91) & 0.01 & 0.87 (0.58 - 1.29) & 0.49 & 0.98 (0.56 - 1.70) & 0.93\\
\hspace{1em}Central & 0.48 (0.31 - 0.74) & 0.00 & 0.58 (0.36 - 0.96) & 0.03 & 0.48 (0.23 - 1.00) & 0.05\\
\hspace{1em}R. Valley & 0.51 (0.35 - 0.74) & 0.00 & 0.79 (0.54 - 1.15) & 0.21 & 0.84 (0.50 - 1.40) & 0.50\\
\hspace{1em}Western & 0.39 (0.26 - 0.59) & 0.00 & 0.48 (0.30 - 0.76) & 0.00 & 0.42 (0.21 - 0.82) & 0.01\\
\hspace{1em}Nyanza & 0.36 (0.24 - 0.54) & 0.00 & 0.52 (0.34 - 0.79) & 0.00 & 0.45 (0.25 - 0.83) & 0.01\\
\addlinespace
\hspace{1em}Nairobi & ref &  & ref &  & ref & \\
\bottomrule
\end{tabular}
\end{table}
\renewcommand{\arraystretch}{1}

Table \ref{tab:sens} summarizes the results based on screening for child
stunting using a household's socioeconomic status, child's age, gender,
and birth order, respectively. The sensitivity of a household's
socioeconomic status as a diagnostic tool for under-five child
malnutrition was 67.4\% (95\% CI: 66.4\% -- 68.4\%), whereas the
specificity was 50.6\% (95\%CI: 50.0\% - 51.1\%). The ability of a
household's socioeconomic status to discriminate between stunted and
non-stunted children was above random guessing (AUC = 0.59; 95\%CI: 0.58
-- 0.60). On the other hand, the sensitivity, specificity, and AUC
values based on screening using the child's age were 49.6\% (95\%CI:
48.5\% - 50.7\%), 52.6\% (95\%CI: 52.0\% - 53.2\%), and 0.51 (95\%CI:
0.50 -- 0.52), respectively. On the other hand, sensitivity and
specificity values based on screening for under-five child stunting
using the child's gender and age were 57.0\% (95\%CI: 55.9\% -- 58.0\%),
51.2\% (95\%CI: 50.6\% - 51.8\%) and 44.9\% (95\%CI: 43.8\% - 46.0\%),
63.8\% (95\%CI: 63.2\% - 64.3\%). The discriminatory ability based on
using the child's age was limited (AUC = 0.51; 95\%CI: 0.50 -- 0.52),
whereas that based on using gender (AUC = 0.54; 95\%CI: 0.53 -- 0.55)
and birth order (AUC = 0.54; 95\%CI: 0.54 -- 0.55) was slightly above
average.

\begin{table}

\caption{\label{tab:sens}Results based on screening for child stunting using a household's socioeconomic status (SES), child's age, gender, and birth order number}
\centering
\begin{tabular}[t]{>{}lcc}
\toprule
\textbf{Variable} & \textbf{Metric} & \textbf{95\% Confidence Interval (CI)}\\
\midrule
\textbf{Household SES} &  & \\
\textbf{} & Sensitivity & 67.4 (66.4 - 68.4)\\
\textbf{} & Specificity & 50.6 (50.0 - 51.1)\\
\textbf{} & AUC & 0.59 (0.58 - 0.60)\\
\textbf{} & NPV & 84.1 (83.5 - 84.6)\\
\addlinespace
\textbf{} & PPV & 28.6 (28.0 - 29.2)\\
\textbf{Child's age} &  & \\
\textbf{} & Sensitivity & 49.6 (48.5 - 50.7)\\
\textbf{} & Specificity & 52.6 (52.0 - 53.2)\\
\textbf{} & AUC & 0.51 (0.50 - 0.52)\\
\addlinespace
\textbf{} & NPV & 78.0 (77.4 - 78.6)\\
\textbf{} & PPV & 23.5 (22.9 - 24.2)\\
\textbf{Child's gender} &  & \\
\textbf{} & Sensitivity & 57.0 (55.9 - 58.0)\\
\textbf{} & Specificity & 51.2 (50.6 - 51.8)\\
\addlinespace
\textbf{} & AUC & 0.54 (0.53 - 0.55)\\
\textbf{} & NPV & 80.2 (79.6 - 80.8)\\
\textbf{} & PPV & 25.5 (24.9 - 26.2)\\
\textbf{Birth order number} &  & \\
\textbf{} & Sensitivity & 44.9 (43.8 - 46.0)\\
\addlinespace
\textbf{} & Specificity & 63.8 (63.2 - 64.3)\\
\textbf{} & AUC & 0.54 (0.54 - 0.55)\\
\textbf{} & NPV & 79.8 (79.2 - 80.3)\\
\textbf{} & PPV & 26.7 (26.0 - 27.5)\\
\bottomrule
\multicolumn{3}{l}{\rule{0pt}{1em}\textit{Note: }}\\
\multicolumn{3}{l}{\rule{0pt}{1em}SES: Socioeconomic status}\\
\multicolumn{3}{l}{\rule{0pt}{1em}AUC: Are under the curve}\\
\multicolumn{3}{l}{\rule{0pt}{1em}NPV: Negative predictive value}\\
\multicolumn{3}{l}{\rule{0pt}{1em}PPV: Positive predictive value.}\\
\end{tabular}
\end{table}

\hypertarget{decomposition-of-the-concentration-indices-for-stunting-and-underweight}{%
\subsection{Decomposition of the concentration indices for stunting and
underweight}\label{decomposition-of-the-concentration-indices-for-stunting-and-underweight}}

In Table \ref{tab:five} we present each determinant of child
malnutrition and its percentage contribution to the observed inequality
in child stunting and underweight for the period 2014 and 2022. We
decomposed the CIs of stunting and underweight which differed
significantly between 2014 and 2022. The percentage contribution of
variables in the Wagstaff decomposition represents the contribution of
the variable to the overall health inequality. Negative values of this
quantity suggest the variable contributes to reductions in overall
inequality in child malnutrition.

A household's socioeconomic status (0.89), maternal education (0.22),
and birth order number (0.16) contributed the most toward the observed
disparities in child stunting in 2014. The contribution of a household's
socioeconomic status increased to 1.7 in 2022, whereas that of a child's
birth order number decreased to 0.06. The contribution of maternal
education to inequality in child stunting disappeared in 2022. In both
2014 and 2022, socioeconomic status (0.96), maternal education (0.58),
and paternal education (0.39) were the largest contributors to
disparities in child underweight. Socioeconomic status is a substantial
factor contributing toward inequality in under-five children stunted and
underweight.

\begin{sidewaystable}[!htpb]

\caption{\label{tab:five}Decomposition of the concentration indices and contributions of determinants of under five child stunting and underweight, 2014 and 2022}
\centering
\begin{tabular}[t]{>{\raggedright\arraybackslash}p{2cm}>{\centering\arraybackslash}p{0.8cm}>{\centering\arraybackslash}p{2cm}>{\centering\arraybackslash}p{0.8cm}>{\centering\arraybackslash}p{2cm}>{\centering\arraybackslash}p{0.8cm}>{\centering\arraybackslash}p{2cm}>{\centering\arraybackslash}p{0.8cm}>{\centering\arraybackslash}p{2cm}}
\toprule
\multicolumn{1}{c}{\textbf{ }} & \multicolumn{4}{c}{\textbf{Stunting}} & \multicolumn{4}{c}{\textbf{Underweight}} \\
\cmidrule(l{3pt}r{3pt}){2-5} \cmidrule(l{3pt}r{3pt}){6-9}
\multicolumn{1}{c}{\textbf{ }} & \multicolumn{2}{c}{\textbf{2014}} & \multicolumn{2}{c}{\textbf{2022}} & \multicolumn{2}{c}{\textbf{2014}} & \multicolumn{2}{c}{\textbf{2022}} \\
\cmidrule(l{3pt}r{3pt}){2-3} \cmidrule(l{3pt}r{3pt}){4-5} \cmidrule(l{3pt}r{3pt}){6-7} \cmidrule(l{3pt}r{3pt}){8-9}
  & CI & $\%$ Contribution & CI & $\%$ Contribution & CI & $\%$ Contribution & CI & $\%$ Contribution\\
\midrule
Childs sex & -0.002 & -0.003 & -0.003 & -0.005 & -0.003 & -0.005 & -0.003 & -0.004\\
Residence & -0.540 & -0.332 & -0.617 & -0.634 & -0.617 & -0.636 & -0.617 & -0.636\\
Religion & -0.050 & -0.007 & -0.019 & -0.001 & -0.020 & -0.018 & -0.020 & -0.018\\
Mothers education & 0.390 & 0.226 & 0.449 & -0.165 & 0.449 & 0.589 & 0.449 & 0.589\\
Mothers age (years) & -0.012 & -0.012 & 0.040 & 0.011 & 0.040 & -0.011 & 0.040 & -0.011\\
\addlinespace
Mothers work & 0.078 & -0.021 & 0.180 & -0.021 & 0.180 & 0.005 & 0.180 & 0.005\\
Fathers education & 0.404 & -0.006 & 0.494 & 0.275 & 0.494 & 0.390 & 0.494 & 0.390\\
Delivery place & 0.332 & 0.046 & 0.181 & -0.051 & 0.181 & 0.003 & 0.181 & 0.003\\
Region & 0.131 & 0.034 & 0.122 & 0.088 & 0.122 & 0.084 & 0.122 & -0.026\\
Birth interval (months) & 0.110 & 0.078 & 0.098 & 0.069 & -0.216 & 0.068 & 0.098 & 0.068\\
\addlinespace
Birth order number & -0.154 & 0.163 & -0.144 & 0.057 & -0.144 & -0.029 & -0.144 & -0.029\\
Childs age (months) & 0.003 & -0.003 & -0.002 & 0.006 & -0.002 & 0.003 & -0.002 & 0.003\\
Wealth index & 0.677 & 0.896 & 0.693 & 1.744 & 0.693 & 0.969 & 0.693 & 0.969\\
\bottomrule
\end{tabular}
\end{sidewaystable}

\hypertarget{sec12}{%
\section{Discussion}\label{sec12}}

Nutrition remains a well-known health measure for children under five
years old. In the current study, we investigated the trends of
socioeconomic inequalities in child malnutrition, determinants, the
utility of these determinants in child malnutrition screening, and the
overall contributions of these determinants to disparities in child
malnutrition in Kenya. Our findings indicate decreases in the prevalence
of all three forms of child malnutrition across the periods examined.
Stunting, the most common of child malnutrition indicators, recorded the
highest decline (8\%) between 2014 and 2022. Analyses by sex of the
child revealed a consistently higher prevalence of all three indicators
among male children compared to females. We found the highest burden of
child malnutrition to be in rural areas despite reports of increases in
poverty levels in urban areas \citep{kenya2018basic}.

The prevalence of child malnutrition was higher in children from
impoverished households and reduced substantially as the socioeconomic
status improved from the poorest to the wealthiest. The improvements in
nutrition status from the poor to the rich can be explained to be the
result of improvements in purchasing power and access to food and
quality healthcare \citep{Abuya2012}. Concentration indices (in absolute
values) for stunting and underweight increased significantly between
2014 and 2022 for stunting and underweight, suggesting that
socioeconomic disparities in these malnutrition indicators worsened
between these years. These analyses revealed a disproportionate effect
of socioeconomic inequality on child malnutrition, where children from
poor economic groups are disadvantaged relative to those from the
wealthiest economic groups. Even though the overall under-five child
malnutrition declined between 2014 and 2022, the inequalities persisted.
This result aligns with previous studies where a disconnect has been
reported between economic endowment and the distribution of these
rewards between economic groups \citep{jonah2018, May2014, mariara2009}.

Our analyses of determinants of under-five child malnutrition revealed
the child's age, birth order number, gender, and household's
socioeconomic status as significant factors associated with an increased
risk of child stunting. These findings conform with those found in
previous studies in low and middle-income countries
\citep{mariara2009, Abuya2012, rabbani2016, jonah2018, kien_trends_2016, Akombi2019}.
Birth order has been implicated as a significant predictor of child
stunting, with higher-order children (\textgreater fifth order) being
more likely to be stunted due to less attention and care from their
parents \citep{rahman2016}. Additionally, with births spaced in quick
succession, high-order children are likely to receive limited lacteal
feeding, a factor that predisposes them to malnutrition
\citep{Gudu2020}. The effect of age can be explained similarly. We
explain the higher risk of stunted growth among male children to be
partly the result of their faster growth rate, which increases their
susceptibility to stunted growth when malnourished, higher biological
fragility \citep{Kraemer2000} and preferential treatment of the female
child \citep{Wamani2007, Farah2019}.

The socioeconomic status of a household is a substantial factor in
determining the nutritional status of children and the inequality in
under-five child malnutrition. A similar result on the contribution of
socioeconomic status of a household to disparities in under-five child
malnutrition has been reported elsewhere in low and middle-income
countries \citep{kien_trends_2016}. The contribution of this factor
appeared to increase in 2022 relative to 2014. The nutritional status of
the Kenyan population has improved as part of the standard development
goals. However, the rapid population growth recorded over the years has
yet to be at par with the rate of growth of the economy. The result has
been that a more significant subset of the population has remained in
poverty \citep{odonnell_analyzing_2008}.

Consequently, the disparities in the population's economic status have
worked to reduce access to essential services for people with low
incomes. While the rich have access to high-quality education,
healthcare, and food, the poor struggle to meet their basic needs
\citep{Masiye2010, Masuku2017, Pathak2011}. With limited finances, a
household's ability to afford a stable supply of food is significantly
reduced, the effects of which include adverse effects on child growth
and cognitive development
\citep{Perignon2014, Bryan2004, Hackett2009, Mutisya2015, Saxena2018}.

Stunting remains a significant indicator of child stunting. In assessing
the nutritional status of children, anthropometric data
--weight-for-height, height-for-age, and weight-for-age -- are often
collected. These data are often prone to random error due to the
complexities involved in their collection in younger children
\citep{Farah2019}. We explored the clinical utility of a household's
socioeconomic status, child's gender, age, and birth order as diagnostic
factors for child stunting. We found that the ability of a household's
socioeconomic status to distinguish stunted from non-stunted children
was higher than that of age, gender, and birth order, with a 67\%
sensitivity. We allude this performance to the significance of
socioeconomic status in determining child health outcomes.

Findings presented in this investigation bear a lot of merits. First, we
used nationally representative population-based survey data to examine
the socioeconomic disparities in child malnutrition and provide crucial
insight into the design of strategies to curb this inequality. These
data present the advantage of relatively large sample sizes and
commendable response rates (\textgreater{} 90\%).

Second, we found that socioeconomic disparities in under-five child
malnutrition have substantially increased since 2014, even though the
overall prevalence of child malnutrition waned. This has been partly due
to commendable improvements in health status and changes in income and
lifestyle, among other factors \citep{kien_trends_2016}. While the
country has experienced economic growth, our findings suggest a
disconnect between this growth and the distribution of wealth in the
general population, with poor households disproportionately affected.
Consequently, inequality in under-five child malnutrition has worsened
even though the overall health indicators have improved.

Third, we decomposed the concentration indices of each determinant of
under-five child malnutrition to decipher the nature of their
contribution to the observed socioeconomic disparities in stunting and
underweight. Our results thus break ground and foster knowledge on
causes and variations in under-five child malnutrition prevalence by
income groups. Fourth, we further explored the utility of significant
malnutrition-related factors as screening tests for chronic malnutrition
and found some potential for socioeconomic status.

The results from this investigation provide pivotal insights for public
health policy and practice. For instance, we found a household's
socioeconomic status to be the largest contributor to health inequality,
with children from poor households being the most vulnerable. These
findings will enable public health policymakers to strategically target
socioeconomic differences to develop effective and thorough childhood
nutrition interventions through policies that improve outcomes for
children from poor households while bridging the gap between rural and
urban development through equitable distribution of resources.

Furthermore, the public health sector can adopt strategies including the
provision of easily accessible medical care, improved sanitation, and
provision of clean drinking water. The government should consider
strategies to reduce migration of people from rural areas to urban areas
through job creation schemes in rural areas, providing child support to
households in poverty, diversifying source of livelihoods for people in
rural areas through provisions of viable alternatives such as commerce,
provision of unemployment benefits as well as taking insurance for the
agricultural sector to enhance food security and enhance equity.

The following limitations should be considered while interpreting the
findings of this study. First, due to reliance on cross-sectional study
data, our findings cannot be construed as suggesting a causal
relationship between socioeconomic indices and child malnutrition.
Second, residence was classified as either urban or rural. Classifying
this variable into these two localities might pose a problem following
the heterogeneity associated with large cities and the unavailability of
data to quantify these dissimilarities. Third, our analysis adjusted for
potentially confounding variables which did not have substantial
missingness and might not include all possible confounders. However, our
employment of standard statistical practices and the included covariates
allow for a robust analysis of the scope of the problem.

\hypertarget{sec13}{%
\section{Conclusion}\label{sec13}}

Understanding the nature of under-five child malnutrition and its
variations by socioeconomic quintiles is crucial in the design of
strategies that target individuals who are most affected by malnutrition
and in keeping the existing disparities under check. Socioeconomic
inequality in under-five child malnutrition increased between 2014 and
2022 despite an overall decrease in prevalence, with children from poor
households disproportionately affected. These inequalities are fueled by
differences in endowments, a significant proportion of which is held by
the household's socioeconomic status. Our exploration of the utility of
a child's age, gender, birth order number, and household's socioeconomic
status as a diagnostic tool for under-five child stunting showed some
potential for a household's socioeconomic status. Despite the economic
growth Kenya has experienced in the recent past, there is a disconnect
between this growth and income distribution between social classes.
Efforts to tackle inequalities in under-five child malnutrition should
target reducing the perceived differences in endowments between the rich
and poor. The Kenyan government should enhance poor households' access
to food, education, and essential resources.

\backmatter

\hypertarget{declarations}{%
\section*{Declarations}\label{declarations}}
\addcontentsline{toc}{section}{Declarations}

\begin{itemize}
\tightlist
\item
  Ethics approval and consent to participate
\end{itemize}

Not applicable

\begin{itemize}
\tightlist
\item
  Consent for publication
\end{itemize}

Not applicable

\begin{itemize}
\tightlist
\item
  Availability of data and materials
\end{itemize}

All data pertaining to the study are available from the Demographic and
Health Survey at \url{https://dhsprogram.com/data/} upon reasonable
request.

\begin{itemize}
\tightlist
\item
  Competing interests
\end{itemize}

The authors have no competing interests to declare.

\begin{itemize}
\tightlist
\item
  Funding
\end{itemize}

Not applicable

\bibliography{bibliography.bib}


\end{document}
